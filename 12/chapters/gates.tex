\renewcommand{\theequation}{\theenumi}
\renewcommand{\thefigure}{\theenumi}
\begin{enumerate}[label=\thesection.\arabic*.,ref=\thesection.\theenumi]
\numberwithin{equation}{enumi}
\numberwithin{figure}{enumi}
\numberwithin{table}{enumi}

\item The following equation is implemented using gates in Fig. \ref{fig:gates_B}
\begin{align}
B=W^{\prime}X+WX^{\prime}Z^{\prime}
\end{align}
\begin{figure}[!ht]
\begin{center}
\resizebox {\columnwidth} {!} {
\begin{karnaugh-map}[4][4][1][][]
    \minterms{1,2,5,6}
    \maxterms{0,3,4,7,8,9,10,11,12,13,14,15}
    \implicant{12}{10}
    \implicant{3}{11}
    \implicant{0}{8}
    % note: position for start of \draw is (0, Y) where Y is
    % the Y size(number of cells high) in this case Y=2
    \draw[color=black, ultra thin] (0,4) --
    node [pos=0.7, above right, anchor=south west] {$XW$} % YOU CAN CHANGE NAME OF VAR HERE, THE $X$ IS USED FOR ITALICS
    node [pos=0.7, below left, anchor=north east] {$ZY$} % SAME FOR THIS
    ++(135:1);
        
    \end{karnaugh-map}

}
\end{center}
\caption{}
\label{fig:gates_B}
\end{figure}
\item The following equation is implemented using gates in Fig. \ref{fig:gates_b}
\begin{align}
b=B^{\prime}+CD+C^{\prime}D^{\prime}
\end{align}
\begin{figure}[!ht]
\begin{center}
\resizebox {\columnwidth} {!} {
\begin{circuitikz}  \draw
(0,2) node[and port] (myand1) {}
(3,0) node[or port] (myor) {}
(0,0) node[american and port](A){}
(-5,2.3) to [short,-*](-3,2.27)
(-5,1.73) to [short,-*](-4,1.73)
(-3,2.27)-|(myand1.in 1)
(-4,1.73)-|(myand1.in 2)
(-3,2.27) --(-3,0.277)-|(A.in 1)
(-4,1.7)--(-4,-0.277)-|(A.in 2)
(0,0)--(2.1,0)
(0.2,2)--(0.2,0.277)-|(myor.in 1)
(-5,-1)--(1.6,-1)-|(myor.in 2)
(-3,0.277) to [short] (-3,0.277)(-3,1.73) to [crossing](-3,1.73) ;

\node at (A.bin 1) [ocirc, left]{} ;
\node at (A.bin 2) [ocirc, left]{} ;
\node at (myor.bin 2)[ocirc, left]{};
\node[left]at(-5,2.3){$C$};
\node[left]at(-5,1.7){$D$};
\node[right]at(3.5,0){$b$};
\node[left]at(-5,-1){$B$};

\end{circuitikz}

}
\end{center}
\caption{}
\label{fig:gates_b}
\end{figure}
\item The following equation is implemented using gates in Fig. \ref{fig:gates_a}
\begin{align}
    a=B^{\prime}D^{\prime}(A^{\prime}C+AC^{\prime})
\end{align}
\begin{figure}[!ht]
\begin{center}
\resizebox {\columnwidth} {!} {




\begin{circuitikz}
 \draw
(0,-1)node[xor port](myxor1){}

(2,2)node[and port](myand2){}

(4,1)node[and port](myand3){}

(-0.5,0.7)node[not port](mynot4){}

(-0.5,2.3)node[not port](mynot5) {}

(myxor1.out)--(myand3.in 2)
(myand2.out)--(myand3.in 1)
(mynot4.out)--(myand2.in 2)
(mynot5.out)--(myand2.in 1);

\node(x3) at (-1.6,-1.3) {$C$};
\node(x2) at (-1.6,-0.7) {$A$}; 
\node(x1) at (-1.6,0.7) {$D$};
\node(x0) at (-1.6,2.3) {$B$};
\node(x3) at (6,1) {$B'D'(A'C+AC')$};

\end{circuitikz}



}
\end{center}
\caption{}
\label{fig:gates_a}
\end{figure}
%
\item The following equation is implemented using gates in Fig. \ref{fig:gates_d}
\begin{align}
d = AB^{\prime}C^{\prime}+A^{\prime}B^{\prime}CD^{\prime}+ABCD^{\prime}
\end{align}
\begin{figure}[!ht]
\begin{center}
\resizebox {\columnwidth} {!} {
\tikzstyle{branch}=[fill,shape=circle,minimum size=3pt,inner sep=0pt]
\begin{tikzpicture}[label distance=2mm]
    \node (x1) at (0,0) {$A$};
    \node (x2) at (0,-1.72) {$B$};
    \node (x3) at (0,-4) {$C$};
    \node (x4) at (0,-5.5) {$D$};
    
    \node[and gate US, draw, logic gate inputs=nnnnnnn] at ($(x1)+(4,-1.65)$) (And1) {};
    \node[and gate US, draw, logic gate inputs=nnnnnnnnn] at ($(x3)+(4,-1)$) (And2) {};
    \node[and gate US, draw, logic gate inputs=nnnnnnnnn] at ($(x4)+(4,-2)$) (And3) {};
    
    \node[or gate US, draw, logic gate inputs=nnnnnnn, anchor=input 1] at ($(And2.output)+(5,0.5)$) (Or1) {};
    
 \draw (x1)--(2.5,0) 
   (2.5,0)--(2.5,-1.3) 
   (2.5,-1.3)--(And1.input 2)
   (x2)--(3.1,-1.64) 
   (3.17,-1.68) node[american] {o}
   (x3)--(2.5,-4) 
   (2.5,-4)--(2.5,-2) 
   (2.5,-2)--(3.14,-2) 
   (3.17,-2)node[american] {o}
   (And1.output)--(8,-1.64)
 
  
   (2.2,0)node[american] {o}   
   (2.2,0)--(2.2,-4.5)
   (2.2,-4.5)--(And2.input 2)
   (2.95,-4.5)node[american] {o}
   (1.6,-1.68)node[american] {o}
   (1.6,-1.68)--(1.6,-4.8)
   (1.6,-4.8)--(And2.input 4)
    (2.95,-4.8)node[american] {o}
    (1,-4)node[american] {o}
    (1,-4)--(1,-5.2)
    (1,-5.2)--(And2.input 6)
    (x4)--(And2.input 8)
     (2.95,-5.5)node[american] {o}
      (And2.output)--(Or1.input 4)
      
      
      
     (2.2,-4.5)--(2.2,-7)
     (2.2,-4.5)node[american] {o}
     (2.2,-7)--(And3.input 2)
     (1.6,-4.8)--(1.6,-7.3)
     (1.6,-7.3)--(And3.input 4)
     (1.6,-4.8)node[american] {o}
     (1,-5.2)--(1,-7.65)
    (1,-5.2)node[american] {o}
    (1,-7.65)--(And3.input 6)
    (0.6,-5.5)--(0.6,-8)
    (0.6,-8)--(And3.input 8)
    (0.6,-5.5)node[american] {o}
    (2.95,-8)node[american] {o}
    (And3.output)--(8,-7.45)
    
    
    (8,-1.64)--(8,-4.7)
    (8,-4.7)--(Or1.input 2)
    (8,-7.45)--(8,-5.35)
    (8,-5.35)--(Or1.input 6)
    (Or1.output)--(13,-5)
     (6,-1.3)node[american] {$AB'C'$}
      (6,-4.5)node[american] {$A'B'CD'$}
       (6,-7)node[american] {$ABCD'$}
     (13,-4)node[american] {$AB'C'+A'B'CD'+ABCD'= d$};
     
\end{tikzpicture}

}
\end{center}
\caption{}
\label{fig:gates_d}
\end{figure}
%
%\item The following equation is implemented using gates in Fig. \ref{fig:gates_e}
%\begin{align}
%%
%e = AD^{\prime}+AB^{\prime}C^{\prime}+B^{\prime}CD^{\prime}
%\end{align}
%\begin{figure}[!ht]
%\begin{center}
%\resizebox {\columnwidth} {!} {
%\input{./figs/gates/e_disp.tex}
%}
%\end{center}
%\caption{}
%\label{fig:gates_e}
%\end{figure}
\end{enumerate}
