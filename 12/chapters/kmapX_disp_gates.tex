\renewcommand{\theequation}{\theenumi}
\renewcommand{\thefigure}{\theenumi}
\begin{enumerate}[label=\thesubsection.\arabic*.,ref=\thesubsection.\theenumi]
\numberwithin{equation}{enumi}
\numberwithin{figure}{enumi}
\numberwithin{table}{enumi}

\item K-Map for $A$: 
The expression in \eqref{eq:inc_A}  can be minimized using the K-map in Fig. \ref{fig:kmap_A}.
In Fig. \ref{fig:kmap_A},  the {\em implicants} in boxes 0,2,4,6 result in $W^{\prime}Z^{\prime}$.  The implicants in
boxes 0,8 result in $W^{\prime}X^{\prime}Y^{\prime}$.  Thus, after minimization using Fig. \ref{eq:kmap_A},  \eqref{eq:inc_A} can be expressed as
%
\begin{equation}
\label{eq:kmap_A}
A = W^{\prime}Z^{\prime}+W^{\prime}X^{\prime}Y^{\prime}
\end{equation}
%
Using the fact that
\begin{align}
\label{eq:inc_Boolean}
\begin{split}
X+X^{\prime} &= 1
\\
XX^{\prime} &= 0,
\end{split}
\end{align}
%
derive \eqref{eq:kmap_A} from \eqref{eq:inc_A} algebraically.
%
%
%
\begin{figure}[!h]
\resizebox {\columnwidth} {!} {
\input{./figs/kmap_A}
}
\caption{K-map for $A$.}
\label{fig:kmap_A}
\end{figure}

\item K-Map for $B$:
From Table \ref{table:counter_decoder}, using boolean logic,
%
\begin{figure}[!h]
\resizebox {\columnwidth} {!} {
\input{./figs/kmap_B}
}
\caption{K-map for $B$.}
\label{fig:kmap_B}
\end{figure}
%
Show that \eqref{eq:inc_B} can be reduced to
\begin{equation}
\label{eq:kmap_B}
B = WX^{\prime}Z^{\prime} + W^{\prime}XZ^{\prime}
\end{equation}
using Fig. \ref{fig:kmap_B}.
\item Derive \eqref{eq:kmap_B} from \eqref{eq:inc_B} algebraically using \eqref{eq:inc_Boolean}.
%
%
\item {K-Map for $C$: }
From Table \ref{table:counter_decoder}, using boolean logic,
%
%
\begin{figure}[!h]
\resizebox {\columnwidth} {!} {
\input{./figs/kmap_C}
}
\caption{K-map for $C$.}
\label{fig:kmap_C}
\end{figure}
%
Show that \eqref{eq:inc_C} can be reduced to
\begin{equation}
\label{eq:kmap_C}
C = WXY^{\prime}Z^{\prime}  +  X^{\prime}YZ^{\prime} + W^{\prime}YZ^{\prime}
\end{equation}
using Fig. \ref{fig:kmap_C}.
%
\item 
Derive \eqref{eq:kmap_C} from \eqref{eq:inc_C} algebraically using \eqref{eq:inc_Boolean}.
%
\item {K-Map for $D$: }
From Table \ref{table:counter_decoder}, using boolean logic,
\begin{equation}
\label{eq:D}
D = WXYZ^{\prime} + W^{\prime}X^{\prime}Y^{\prime}Z
\end{equation}
%
\begin{figure}[!h]
\resizebox {\columnwidth} {!} {
\input{./figs/kmap_D}
}
\caption{K-map for $D$.}
\label{fig:kmap_D}
\end{figure}
%
\item 
Minimize \eqref{eq:D} using Fig. \ref{fig:kmap_D}.
%
\item Modify your C program to verify the 
%Download the code in
%\begin{lstlisting}
%wget https://raw.githubusercontent.com/gadepall/arduino/master/7447/codes/inc_dec/inc_dec.ino
%\end{lstlisting}
%
%and modify it using 
the K-Map equations for A,B,C and D in \eqref{eq:kmap_A}, \eqref{eq:kmap_A}, \eqref{eq:kmap_A}
and \eqref{eq:kmap_A} respectively.
%Execute and verify.
\item Revise by using don't care conditions and verify through a C code.
%\item {Display Decoder:}
%%is the truth table for the display decoder in Fig. \ref{fig:dec_counter}.  
%Use K-maps to obtain the minimized expressions for $a,b,c,d,e,f,g$ in terms of $A,B,C,D$ in Table \ref{table:disp_dec} 
%with and without don't care conditions.
%%
%\input{./figs/disp_dec}
\end{enumerate}
%
%
