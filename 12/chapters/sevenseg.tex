%\subsection{Introduction}
\renewcommand{\theequation}{\theenumi}
\renewcommand{\thefigure}{\theenumi}
\begin{enumerate}[label=\thesection.\arabic*.,ref=\thesection.\theenumi]
\numberwithin{equation}{enumi}
\numberwithin{figure}{enumi}
\numberwithin{table}{enumi}

\item  Fig. \ref{fig:sevenseg} shows a seven segment display with pins $a,b,c,d,e,f,g$.  Each of these pins is connected to an LED (light emitting device).

\begin{figure}[!ht]
\centering
\resizebox {\columnwidth} {!} {
%\subsection{Introduction}
\renewcommand{\theequation}{\theenumi}
\renewcommand{\thefigure}{\theenumi}
\begin{enumerate}[label=\thesection.\arabic*.,ref=\thesection.\theenumi]
\numberwithin{equation}{enumi}
\numberwithin{figure}{enumi}
\numberwithin{table}{enumi}

\item  Fig. \ref{fig:sevenseg} shows a seven segment display with pins $a,b,c,d,e,f,g$.  Each of these pins is connected to an LED (light emitting device).

\begin{figure}[!ht]
\centering
\resizebox {\columnwidth} {!} {
%\subsection{Introduction}
\renewcommand{\theequation}{\theenumi}
\renewcommand{\thefigure}{\theenumi}
\begin{enumerate}[label=\thesection.\arabic*.,ref=\thesection.\theenumi]
\numberwithin{equation}{enumi}
\numberwithin{figure}{enumi}
\numberwithin{table}{enumi}

\item  Fig. \ref{fig:sevenseg} shows a seven segment display with pins $a,b,c,d,e,f,g$.  Each of these pins is connected to an LED (light emitting device).

\begin{figure}[!ht]
\centering
\resizebox {\columnwidth} {!} {
%\subsection{Introduction}
\renewcommand{\theequation}{\theenumi}
\renewcommand{\thefigure}{\theenumi}
\begin{enumerate}[label=\thesection.\arabic*.,ref=\thesection.\theenumi]
\numberwithin{equation}{enumi}
\numberwithin{figure}{enumi}
\numberwithin{table}{enumi}

\item  Fig. \ref{fig:sevenseg} shows a seven segment display with pins $a,b,c,d,e,f,g$.  Each of these pins is connected to an LED (light emitting device).

\begin{figure}[!ht]
\centering
\resizebox {\columnwidth} {!} {
\input{./figs/sevenseg.tex}
}
\caption{}
\label{fig:sevenseg}
\end{figure}

\item Fig. \ref{fig:sevenseg12} shows how to generate the numbers on the display using Table
\ref{table:arduioport}.  Complete Table \ref{table:arduioport} by drawing the figures for all numbers from 0-9.

\begin{figure}[!h]
\begin{center}
\resizebox {\columnwidth} {!} {
\input{./figs/sevenseg12.tex}
}
\end{center}
\caption{}
\label{fig:sevenseg12}
\end{figure}


\begin{table}[!h]
\centering
\input{./tables/arduinoport.tex}
\caption{}
\label{table:arduioport}
\end{table}


\end{enumerate}

}
\caption{}
\label{fig:sevenseg}
\end{figure}

\item Fig. \ref{fig:sevenseg12} shows how to generate the numbers on the display using Table
\ref{table:arduioport}.  Complete Table \ref{table:arduioport} by drawing the figures for all numbers from 0-9.

\begin{figure}[!h]
\begin{center}
\resizebox {\columnwidth} {!} {
\input{./figs/sevenseg12.tex}
}
\end{center}
\caption{}
\label{fig:sevenseg12}
\end{figure}


\begin{table}[!h]
\centering
\input{./tables/arduinoport.tex}
\caption{}
\label{table:arduioport}
\end{table}


\end{enumerate}

}
\caption{}
\label{fig:sevenseg}
\end{figure}

\item Fig. \ref{fig:sevenseg12} shows how to generate the numbers on the display using Table
\ref{table:arduioport}.  Complete Table \ref{table:arduioport} by drawing the figures for all numbers from 0-9.

\begin{figure}[!h]
\begin{center}
\resizebox {\columnwidth} {!} {
\input{./figs/sevenseg12.tex}
}
\end{center}
\caption{}
\label{fig:sevenseg12}
\end{figure}


\begin{table}[!h]
\centering
\input{./tables/arduinoport.tex}
\caption{}
\label{table:arduioport}
\end{table}


\end{enumerate}

}
\caption{}
\label{fig:sevenseg}
\end{figure}

\item Fig. \ref{fig:sevenseg12} shows how to generate the numbers on the display using Table
\ref{table:arduioport}.  Complete Table \ref{table:arduioport} by drawing the figures for all numbers from 0-9.

\begin{figure}[!h]
\begin{center}
\resizebox {\columnwidth} {!} {
\input{./figs/sevenseg12.tex}
}
\end{center}
\caption{}
\label{fig:sevenseg12}
\end{figure}


\begin{table}[!h]
\centering
\input{./tables/arduinoport.tex}
\caption{}
\label{table:arduioport}
\end{table}


\end{enumerate}
