\subsection{प्रस्तावना}
दशक गणित्र  एक  तुलयात्मक  परिपथ है जो अनुक्रम 0-9 की समान अतिकाल से निरन्तर   गणना करता है।  इसका खंड आरेख 
आकृति. \ref{fig:dec_counter} में उपलब्ध है।  

\begin{figure}[!h]
\resizebox {\columnwidth} {!} {
%\documentclass{article}

%\usepackage[latin1]{inputenc}
%\usepackage{tikz}
%\usetikzlibrary{shapes,arrows}

%%%%<
%\usepackage{verbatim}
%\usepackage[active,tightpage]{preview}
%\PreviewEnvironment{tikzpicture}
%\setlength\PreviewBorder{5pt}%
%%%%>

%\begin{comment}
%:Title: Simple flow chart
%:Tags: Diagrams

%With PGF/TikZ you can draw flow charts with relative ease. This flow chart from [1]_
%outlines an algorithm for identifying the parameters of an autonomous underwater vehicle model. 

%Note that relative node
%placement has been used to avoid placing nodes explicitly. This feature was
%introduced in PGF/TikZ >= 1.09.

%.. [1] Bossley, K.; Brown, M. & Harris, C. Neurofuzzy identification of an autonomous underwater vehicle `International Journal of Systems Science`, 1999, 30, 901-913 


%\end{comment}


%\begin{document}
%\pagestyle{empty}


% Define block styles
\tikzstyle{decision} = [diamond, draw, fill=blue!20, 
    text width=4.5em, text badly centered, node distance=3cm, inner sep=0pt]
%\tikzstyle{block} = [rectangle, draw, fill=blue!20, 
%    text width=5em, text centered, rounded corners, minimum height=4em]
\tikzstyle{block} = [rectangle, draw, 
    text width=5em, text centered, rounded corners, minimum height=4em]

\tikzstyle{line} = [draw, -latex']
\tikzstyle{cloud} = [draw, ellipse,fill=red!20, node distance=3cm,
    minimum height=2em]
    
\begin{tikzpicture}[node distance = 3cm, auto]
    % Place nodes
%    \node [block] (init) {Incrementing Decoder};
    \node [block] (init) {परवर्ती गूढ़वाचक};
%    \node [cloud, left of=init] (expert) {expert};
%    \node [cloud, right of=init] (system) {system};
%    \node [block, below of=init, node distance = 4cm] (identify) {Display Decoder};
    \node [block, below of=init, node distance = 4cm] (identify) {प्रदर्शी गूढ़वाचक};
%    \node [block, below of=identify ] (evaluate) {Seven-Segment Display};
    \node [block, below of=identify ] (evaluate) {सप्तांश प्रदर्शी};
%    \node [block, right of=identify, node distance = 4cm] (delay) {Delay};
     %\node [block, (4,-3)] (q1) {Delay};
%	\node at (4,-2)[block] (delay) {Delay};
	\node at (4,-2)[block] (delay) {अतिकाल};
\begin{scope}[->,>=latex]
    \foreach \i in {-3,-1,1,3}
    { 
%      \draw[->] ([yshift=\i * 0.2 cm]identify.east) -- ([yshift=\i * 0.2 cm]delay.west) ;
      \draw[->] ([xshift=\i * 0.2 cm]delay.north) |- ([yshift=\i * 0.2 cm]init.east) ;
      \draw[->] ([xshift=\i * 0.2 cm]init.south) -- ([xshift=\i * 0.2 cm]identify.north) ;
       \draw node at (\i * 0.2,-2+\i * 0.2) { \textbullet} ;
       \draw[->] (\i * 0.2,-2+\i * 0.2) -- ([yshift=\i * 0.2 cm]delay.west) ;
      
    }
\foreach \i in {-3,...,3}
    { 
      \draw[->] ([xshift=\i * 0.35 cm]identify.south) -- ([xshift=\i * 0.35 cm]evaluate.north) ;
    }
\foreach [count=\i] \j in {a,b,...,g}{
            \node (\i) at ( 1.6-\i * 0.35, -5.5) {\j} ;
            }
\foreach [count=\i] \j in {A,B,C,D}{
            \node (\i) at ( 0.8-\i * 0.4, -1.0-\i*0.4) {\j} ;
            }

\foreach [count=\i] \j in {W,X,Y,Z}{
            \node (\i) at ( 1.6, 1.2-\i*0.4) {\j} ;
            }
    
\end{scope}

 %   \node [block, left of=evaluate, node distance=3cm] (update) {update model};
  %  \node [decision, below of=evaluate] (decide) {is best candidate better?};
%    \node [block, below of=decide, node distance=3cm] (stop) {stop};
    % Draw edges
%    \path [line] (init) -- (identify);
    \path [line] (identify) -- (evaluate);
%    \path [line] (evaluate) -- (decide);
  %  \path [line] (decide) -| node [near start] {yes} (update);
   % \path [line] (update) |- (identify);
 %   \path [line] (decide) -- node {no}(stop);
%    \path [line,dashed] (expert) -- (init);
%    \path [line,dashed] (system) -- (init);
%    \path [line,dashed] (system) |- (evaluate);
\end{tikzpicture}
%}

%\end{document}

%\subsection{प्रस्तावना}
दशक गणित्र  एक  तुलयात्मक  परिपथ है जो अनुक्रम 0-9 की समान अतिकाल से निरन्तर   गणना करता है।  इसका खंड आरेख 
आकृति. \ref{fig:dec_counter} में उपलब्ध है।  

\begin{figure}[!h]
\resizebox {\columnwidth} {!} {
%\documentclass{article}

%\usepackage[latin1]{inputenc}
%\usepackage{tikz}
%\usetikzlibrary{shapes,arrows}

%%%%<
%\usepackage{verbatim}
%\usepackage[active,tightpage]{preview}
%\PreviewEnvironment{tikzpicture}
%\setlength\PreviewBorder{5pt}%
%%%%>

%\begin{comment}
%:Title: Simple flow chart
%:Tags: Diagrams

%With PGF/TikZ you can draw flow charts with relative ease. This flow chart from [1]_
%outlines an algorithm for identifying the parameters of an autonomous underwater vehicle model. 

%Note that relative node
%placement has been used to avoid placing nodes explicitly. This feature was
%introduced in PGF/TikZ >= 1.09.

%.. [1] Bossley, K.; Brown, M. & Harris, C. Neurofuzzy identification of an autonomous underwater vehicle `International Journal of Systems Science`, 1999, 30, 901-913 


%\end{comment}


%\begin{document}
%\pagestyle{empty}


% Define block styles
\tikzstyle{decision} = [diamond, draw, fill=blue!20, 
    text width=4.5em, text badly centered, node distance=3cm, inner sep=0pt]
%\tikzstyle{block} = [rectangle, draw, fill=blue!20, 
%    text width=5em, text centered, rounded corners, minimum height=4em]
\tikzstyle{block} = [rectangle, draw, 
    text width=5em, text centered, rounded corners, minimum height=4em]

\tikzstyle{line} = [draw, -latex']
\tikzstyle{cloud} = [draw, ellipse,fill=red!20, node distance=3cm,
    minimum height=2em]
    
\begin{tikzpicture}[node distance = 3cm, auto]
    % Place nodes
%    \node [block] (init) {Incrementing Decoder};
    \node [block] (init) {परवर्ती गूढ़वाचक};
%    \node [cloud, left of=init] (expert) {expert};
%    \node [cloud, right of=init] (system) {system};
%    \node [block, below of=init, node distance = 4cm] (identify) {Display Decoder};
    \node [block, below of=init, node distance = 4cm] (identify) {प्रदर्शी गूढ़वाचक};
%    \node [block, below of=identify ] (evaluate) {Seven-Segment Display};
    \node [block, below of=identify ] (evaluate) {सप्तांश प्रदर्शी};
%    \node [block, right of=identify, node distance = 4cm] (delay) {Delay};
     %\node [block, (4,-3)] (q1) {Delay};
%	\node at (4,-2)[block] (delay) {Delay};
	\node at (4,-2)[block] (delay) {अतिकाल};
\begin{scope}[->,>=latex]
    \foreach \i in {-3,-1,1,3}
    { 
%      \draw[->] ([yshift=\i * 0.2 cm]identify.east) -- ([yshift=\i * 0.2 cm]delay.west) ;
      \draw[->] ([xshift=\i * 0.2 cm]delay.north) |- ([yshift=\i * 0.2 cm]init.east) ;
      \draw[->] ([xshift=\i * 0.2 cm]init.south) -- ([xshift=\i * 0.2 cm]identify.north) ;
       \draw node at (\i * 0.2,-2+\i * 0.2) { \textbullet} ;
       \draw[->] (\i * 0.2,-2+\i * 0.2) -- ([yshift=\i * 0.2 cm]delay.west) ;
      
    }
\foreach \i in {-3,...,3}
    { 
      \draw[->] ([xshift=\i * 0.35 cm]identify.south) -- ([xshift=\i * 0.35 cm]evaluate.north) ;
    }
\foreach [count=\i] \j in {a,b,...,g}{
            \node (\i) at ( 1.6-\i * 0.35, -5.5) {\j} ;
            }
\foreach [count=\i] \j in {A,B,C,D}{
            \node (\i) at ( 0.8-\i * 0.4, -1.0-\i*0.4) {\j} ;
            }

\foreach [count=\i] \j in {W,X,Y,Z}{
            \node (\i) at ( 1.6, 1.2-\i*0.4) {\j} ;
            }
    
\end{scope}

 %   \node [block, left of=evaluate, node distance=3cm] (update) {update model};
  %  \node [decision, below of=evaluate] (decide) {is best candidate better?};
%    \node [block, below of=decide, node distance=3cm] (stop) {stop};
    % Draw edges
%    \path [line] (init) -- (identify);
    \path [line] (identify) -- (evaluate);
%    \path [line] (evaluate) -- (decide);
  %  \path [line] (decide) -| node [near start] {yes} (update);
   % \path [line] (update) |- (identify);
 %   \path [line] (decide) -- node {no}(stop);
%    \path [line,dashed] (expert) -- (init);
%    \path [line,dashed] (system) -- (init);
%    \path [line,dashed] (system) |- (evaluate);
\end{tikzpicture}
%}

%\end{document}

%\subsection{प्रस्तावना}
दशक गणित्र  एक  तुलयात्मक  परिपथ है जो अनुक्रम 0-9 की समान अतिकाल से निरन्तर   गणना करता है।  इसका खंड आरेख 
आकृति. \ref{fig:dec_counter} में उपलब्ध है।  

\begin{figure}[!h]
\resizebox {\columnwidth} {!} {
%\documentclass{article}

%\usepackage[latin1]{inputenc}
%\usepackage{tikz}
%\usetikzlibrary{shapes,arrows}

%%%%<
%\usepackage{verbatim}
%\usepackage[active,tightpage]{preview}
%\PreviewEnvironment{tikzpicture}
%\setlength\PreviewBorder{5pt}%
%%%%>

%\begin{comment}
%:Title: Simple flow chart
%:Tags: Diagrams

%With PGF/TikZ you can draw flow charts with relative ease. This flow chart from [1]_
%outlines an algorithm for identifying the parameters of an autonomous underwater vehicle model. 

%Note that relative node
%placement has been used to avoid placing nodes explicitly. This feature was
%introduced in PGF/TikZ >= 1.09.

%.. [1] Bossley, K.; Brown, M. & Harris, C. Neurofuzzy identification of an autonomous underwater vehicle `International Journal of Systems Science`, 1999, 30, 901-913 


%\end{comment}


%\begin{document}
%\pagestyle{empty}


% Define block styles
\tikzstyle{decision} = [diamond, draw, fill=blue!20, 
    text width=4.5em, text badly centered, node distance=3cm, inner sep=0pt]
%\tikzstyle{block} = [rectangle, draw, fill=blue!20, 
%    text width=5em, text centered, rounded corners, minimum height=4em]
\tikzstyle{block} = [rectangle, draw, 
    text width=5em, text centered, rounded corners, minimum height=4em]

\tikzstyle{line} = [draw, -latex']
\tikzstyle{cloud} = [draw, ellipse,fill=red!20, node distance=3cm,
    minimum height=2em]
    
\begin{tikzpicture}[node distance = 3cm, auto]
    % Place nodes
%    \node [block] (init) {Incrementing Decoder};
    \node [block] (init) {परवर्ती गूढ़वाचक};
%    \node [cloud, left of=init] (expert) {expert};
%    \node [cloud, right of=init] (system) {system};
%    \node [block, below of=init, node distance = 4cm] (identify) {Display Decoder};
    \node [block, below of=init, node distance = 4cm] (identify) {प्रदर्शी गूढ़वाचक};
%    \node [block, below of=identify ] (evaluate) {Seven-Segment Display};
    \node [block, below of=identify ] (evaluate) {सप्तांश प्रदर्शी};
%    \node [block, right of=identify, node distance = 4cm] (delay) {Delay};
     %\node [block, (4,-3)] (q1) {Delay};
%	\node at (4,-2)[block] (delay) {Delay};
	\node at (4,-2)[block] (delay) {अतिकाल};
\begin{scope}[->,>=latex]
    \foreach \i in {-3,-1,1,3}
    { 
%      \draw[->] ([yshift=\i * 0.2 cm]identify.east) -- ([yshift=\i * 0.2 cm]delay.west) ;
      \draw[->] ([xshift=\i * 0.2 cm]delay.north) |- ([yshift=\i * 0.2 cm]init.east) ;
      \draw[->] ([xshift=\i * 0.2 cm]init.south) -- ([xshift=\i * 0.2 cm]identify.north) ;
       \draw node at (\i * 0.2,-2+\i * 0.2) { \textbullet} ;
       \draw[->] (\i * 0.2,-2+\i * 0.2) -- ([yshift=\i * 0.2 cm]delay.west) ;
      
    }
\foreach \i in {-3,...,3}
    { 
      \draw[->] ([xshift=\i * 0.35 cm]identify.south) -- ([xshift=\i * 0.35 cm]evaluate.north) ;
    }
\foreach [count=\i] \j in {a,b,...,g}{
            \node (\i) at ( 1.6-\i * 0.35, -5.5) {\j} ;
            }
\foreach [count=\i] \j in {A,B,C,D}{
            \node (\i) at ( 0.8-\i * 0.4, -1.0-\i*0.4) {\j} ;
            }

\foreach [count=\i] \j in {W,X,Y,Z}{
            \node (\i) at ( 1.6, 1.2-\i*0.4) {\j} ;
            }
    
\end{scope}

 %   \node [block, left of=evaluate, node distance=3cm] (update) {update model};
  %  \node [decision, below of=evaluate] (decide) {is best candidate better?};
%    \node [block, below of=decide, node distance=3cm] (stop) {stop};
    % Draw edges
%    \path [line] (init) -- (identify);
    \path [line] (identify) -- (evaluate);
%    \path [line] (evaluate) -- (decide);
  %  \path [line] (decide) -| node [near start] {yes} (update);
   % \path [line] (update) |- (identify);
 %   \path [line] (decide) -- node {no}(stop);
%    \path [line,dashed] (expert) -- (init);
%    \path [line,dashed] (system) -- (init);
%    \path [line,dashed] (system) |- (evaluate);
\end{tikzpicture}
%}

%\end{document}

%\subsection{प्रस्तावना}
दशक गणित्र  एक  तुलयात्मक  परिपथ है जो अनुक्रम 0-9 की समान अतिकाल से निरन्तर   गणना करता है।  इसका खंड आरेख 
आकृति. \ref{fig:dec_counter} में उपलब्ध है।  

\begin{figure}[!h]
\resizebox {\columnwidth} {!} {
\input{./figs/decade/decade_counter_hindi}
%\input{./figs/decade/decade_counter}
}
\caption{दशक गणित्र का खंड आरेख}
\label{fig:dec_counter}
\end{figure}
%
\subsection{परिमित अवस्था यंत्र}
%
\ref{fig:dec_counter} का परिमित अवस्था यंत्र आरेख आकृति. \ref{fig:fsm_counter} में उपलब्ध है।  $s_0$ वह अवस्था है जहां परवर्ती गूढ़वाचक का आगत मूल्य  0 है।  दशक गणित्र  की अवस्थान्तरण सारणी   \ref{table:counter_decoder}  में प्रस्तुत हैं।  इसमें वर्तमान अवस्था चर  का प्रबोधन  $W,X,Y,Z$ से है एवं आगामी अवस्था चर  $A,B,C,D$ द्वारा प्रबोधित है।
\begin{figure}[!h]
\centering
\resizebox {\columnwidth} {!} {
\input{./figs/decade/fsm_counter_hindi}
}
\caption{दशक गणित्र की अवस्थायें।}
\label{fig:fsm_counter}
\end{figure}
%
\subsection{परिमित अवस्था यंत्र के द्वारा दशक गणित्र का अभिकल्प}

 आकृति. \ref{fig:dff} में D-द्विविध के व्यूह से दशक गणित्र का अभिकल्प उपलब्ध है।  यहाँ D-द्विविध आकृति \ref{fig:dec_counter} में अतिकाल खंड को  आकृति \ref{fig:dff} के परिपथ में कार्यान्वित करता है। D-द्विविध आगत मूल्य को घड़ी के आवर्त समय के पश्चात निर्गमन
करता है।
\begin{figure}[!h]
\resizebox {\columnwidth} {!} {
\input{./figs/decade/dff_hindi}
}
\caption{D-द्विविध द्वारा परिमित अवस्था यंत्र का कार्यान्वयन।}
\label{fig:dff}
\end{figure}
%
आकृति \ref{fig:dff} में यंत्रोपवस्तु मूल्य  सारणी \ref{table:fsm_counter} में प्रदत्त है।
\begin{table}[!h]
\resizebox {\columnwidth} {!} {
\input{./tables/flip_flop_hindi}
}
\caption{यंत्रोपवस्तु मूल्य।}
\label{table:fsm_counter}
\end{table}

उपरोक्त विधान से पूर्ववर्ती गूढ़वाचक का अभिकल्प करें।

%\begin{equation}
%\text{No. of D Flip-Flops} = \ceil{\log_{2}\brak{\text{No. of States}}}
%\end{equation}
%For the FSM in Fig. \ref{fig:fsm_counter}, the number of states is 9, hence the number flipflops required = 4.  
%\begin{problem}
%Design a decade down counter (counts from 9 to 0 repeatedly) using an FSM.  
%\end{problem}

\end{document}



}
\caption{दशक गणित्र का खंड आरेख}
\label{fig:dec_counter}
\end{figure}
%
\subsection{परिमित अवस्था यंत्र}
%
\ref{fig:dec_counter} का परिमित अवस्था यंत्र आरेख आकृति. \ref{fig:fsm_counter} में उपलब्ध है।  $s_0$ वह अवस्था है जहां परवर्ती गूढ़वाचक का आगत मूल्य  0 है।  दशक गणित्र  की अवस्थान्तरण सारणी   \ref{table:counter_decoder}  में प्रस्तुत हैं।  इसमें वर्तमान अवस्था चर  का प्रबोधन  $W,X,Y,Z$ से है एवं आगामी अवस्था चर  $A,B,C,D$ द्वारा प्रबोधित है।
\begin{figure}[!h]
\centering
\resizebox {\columnwidth} {!} {
\usetikzlibrary{arrows,automata, positioning, calc}
%\usetikzlibrary{arrows,automata, calc}
%\begin{tikzpicture}[->,shorten >=1pt,node distance=2cm,on grid,auto] 
\begin{tikzpicture}[->,auto] 
   \node[ ] (s_00)   {}; 
   \foreach \i [count=\ni from 1] in {36,72,...,324}
%       \node[state] (s_\ni) [above right = {2*sin(\i)} and {2*(cos(\i)} of s_00]  {\ni};
        \node[state] (s_\ni) [above right = {2*sin(\i)} and {2*(cos(\i)} of s_00]  {$s_{\ni}$};        
        
        \node[state,initial] (s_0) [above right = {0} and {2} of s_00]  {$s_0$};     

   \foreach \i  [count=\j from 1] in {0,1,...,8}
		\path	(s_\i) edge [bend right]  (s_\j) ;

		\path	(s_9) edge [bend right]  (s_0) ;		
           
\end{tikzpicture}

}
\caption{दशक गणित्र की अवस्थायें।}
\label{fig:fsm_counter}
\end{figure}
%
\subsection{परिमित अवस्था यंत्र के द्वारा दशक गणित्र का अभिकल्प}

 आकृति. \ref{fig:dff} में D-द्विविध के व्यूह से दशक गणित्र का अभिकल्प उपलब्ध है।  यहाँ D-द्विविध आकृति \ref{fig:dec_counter} में अतिकाल खंड को  आकृति \ref{fig:dff} के परिपथ में कार्यान्वित करता है। D-द्विविध आगत मूल्य को घड़ी के आवर्त समय के पश्चात निर्गमन
करता है।
\begin{figure}[!h]
\resizebox {\columnwidth} {!} {
%\documentclass{standalone}
%\usepackage{pgf,tikz}
%\usetikzlibrary{calc,arrows}
%\usepackage{amsmath}

\makeatletter

% Data Flip Flip (DFF) shape
\pgfdeclareshape{dff}{
  % The 'minimum width' and 'minimum height' keys, not the content, determine
  % the size
  \savedanchor\northeast{%
    \pgfmathsetlength\pgf@x{\pgfshapeminwidth}%
    \pgfmathsetlength\pgf@y{\pgfshapeminheight}%
    \pgf@x=0.5\pgf@x
    \pgf@y=0.5\pgf@y
  }
  % This is redundant, but makes some things easier:
  \savedanchor\southwest{%
    \pgfmathsetlength\pgf@x{\pgfshapeminwidth}%
    \pgfmathsetlength\pgf@y{\pgfshapeminheight}%
    \pgf@x=-0.5\pgf@x
    \pgf@y=-0.5\pgf@y
  }
  % Inherit from rectangle
  \inheritanchorborder[from=rectangle]

  % Define same anchor a normal rectangle has
  \anchor{center}{\pgfpointorigin}
  \anchor{north}{\northeast \pgf@x=0pt}
  \anchor{east}{\northeast \pgf@y=0pt}
  \anchor{south}{\southwest \pgf@x=0pt}
  \anchor{west}{\southwest \pgf@y=0pt}
  \anchor{north east}{\northeast}
  \anchor{north west}{\northeast \pgf@x=-\pgf@x}
  \anchor{south west}{\southwest}
  \anchor{south east}{\southwest \pgf@x=-\pgf@x}
  \anchor{text}{
    \pgfpointorigin
    \advance\pgf@x by -.5\wd\pgfnodeparttextbox%
    \advance\pgf@y by -.5\ht\pgfnodeparttextbox%
    \advance\pgf@y by +.5\dp\pgfnodeparttextbox%
  }

  % Define anchors for signal ports
  \anchor{D}{
    \pgf@process{\northeast}%
    \pgf@x=-1\pgf@x%
    \pgf@y=.5\pgf@y%
  }
  \anchor{CLK}{
    \pgf@process{\northeast}%
    \pgf@x=-1\pgf@x%
    \pgf@y=-.66666\pgf@y%
  }
  \anchor{CE}{
    \pgf@process{\northeast}%
    \pgf@x=-1\pgf@x%
    \pgf@y=-0.33333\pgf@y%
  }
  \anchor{Q}{
    \pgf@process{\northeast}%
    \pgf@y=.5\pgf@y%
  }
  \anchor{Qn}{
    \pgf@process{\northeast}%
    \pgf@y=-.5\pgf@y%
  }
  \anchor{R}{
    \pgf@process{\northeast}%
    \pgf@x=0pt%
  }
  \anchor{S}{
    \pgf@process{\northeast}%
    \pgf@x=0pt%
    \pgf@y=-\pgf@y%
  }
  % Draw the rectangle box and the port labels
  \backgroundpath{
    % Rectangle box
    \pgfpathrectanglecorners{\southwest}{\northeast}
    % Angle (>) for clock input
    \pgf@anchor@dff@CLK
    \pgf@xa=\pgf@x \pgf@ya=\pgf@y
    \pgf@xb=\pgf@x \pgf@yb=\pgf@y
    \pgf@xc=\pgf@x \pgf@yc=\pgf@y
    \pgfmathsetlength\pgf@x{1.6ex} % size depends on font size
    \advance\pgf@ya by \pgf@x
    \advance\pgf@xb by \pgf@x
    \advance\pgf@yc by -\pgf@x
    \pgfpathmoveto{\pgfpoint{\pgf@xa}{\pgf@ya}}
    \pgfpathlineto{\pgfpoint{\pgf@xb}{\pgf@yb}}
    \pgfpathlineto{\pgfpoint{\pgf@xc}{\pgf@yc}}
    \pgfclosepath

    % Draw port labels
    \begingroup
    \tikzset{flip flop/port labels} % Use font from this style
    \tikz@textfont

    \pgf@anchor@dff@D
    \pgftext[left,base,at={\pgfpoint{\pgf@x}{\pgf@y}},x=\pgfshapeinnerxsep]{\raisebox{-0.75ex}{D}}

    \pgf@anchor@dff@CE
    \pgftext[left,base,at={\pgfpoint{\pgf@x}{\pgf@y}},x=\pgfshapeinnerxsep]{\raisebox{-0.75ex}{CE}}

    \pgf@anchor@dff@Q
    \pgftext[right,base,at={\pgfpoint{\pgf@x}{\pgf@y}},x=-\pgfshapeinnerxsep]{\raisebox{-.75ex}{Q}}

    \pgf@anchor@dff@Qn
    \pgftext[right,base,at={\pgfpoint{\pgf@x}{\pgf@y}},x=-\pgfshapeinnerxsep]{\raisebox{-.75ex}{$\overline{\mbox{Q}}$}}

    \pgf@anchor@dff@R
    \pgftext[top,at={\pgfpoint{\pgf@x}{\pgf@y}},y=-\pgfshapeinnerysep]{R}

    \pgf@anchor@dff@S
    \pgftext[bottom,at={\pgfpoint{\pgf@x}{\pgf@y}},y=\pgfshapeinnerysep]{S}
    \endgroup
  }
}

% Key to add font macros to the current font
\tikzset{add font/.code={\expandafter\def\expandafter\tikz@textfont\expandafter{\tikz@textfont#1}}} 

% Define default style for this node
\tikzset{flip flop/port labels/.style={font=\sffamily\scriptsize}}
\tikzset{every dff node/.style={draw,minimum width=2cm,minimum 
height=2.828427125cm,very thick,inner sep=1mm,outer sep=0pt,cap=round,add 
font=\sffamily}}
\tikzstyle{block} = [rectangle, draw, 
    text width=5em, text centered, rounded corners, minimum height=4em]


%\makeatother

%\begin{document}

\begin{tikzpicture}[font=\sffamily,>=triangle 45]
  \def\N{3}  % Number of Flip-Flops minus one

  % Place FFs
    \foreach \i [count=\m from 0] in {A,B,C,D}  
       \node [shape=dff] (DFF\m) at ($ 3*(0,\m) $) {\i};
%  \foreach \m in {0,...,\N}
%    \node [shape=dff] (DFF\m) at ($ 3*(0,\m) $) {Bit \#\m};

%  \def\N{7}  % Number of Flip-Flops minus one
%
%  % Place FFs
%  \foreach \m in {0,...,\N}
%    \node [shape=dff] (DFF\m) at ($ 3*(\m,0) $) {Bit \#\m};
%
%  % Connect FFs (Q1 with D1, etc.)
%  \def\p{0}  % Used to save the previous number
%  \foreach \m in {1,...,\N} { % Note that it starts with 1, not 0
%    \draw [->] (DFF\p.Q) -- (DFF\m.D);
%    \global\let\p\m
%  }
%
  % Connect and label data in- and output port
%  \draw [<-] (DFF0.D) -- +(-1,0) node [anchor=east] {input} ;
%  \draw [->] (DFF\N.Q) -- +(1,0) node [anchor=west] {output};
%
%  % 'Reset' port label
%  \path (DFF0) +(-2cm,+2cm) coordinate (temp)
%    node [anchor=east] {reset};
%  % Connect resets
%  \foreach \m in {0,...,\N}
%    \draw [->] (temp) -| (DFF\m.R);
%
%  % 'Set' port label
%  \path (DFF0) +(-2cm,-2cm) coordinate (temp)
%    node [anchor=east] {set};
%  % Connect sets
%  \foreach \m in {0,...,\N}
%    \draw [->] (temp) -| (DFF\m.S);
%
  % Clock port label
  \path (DFF0) +(-2cm,-2.5cm) coordinate (temp)
    node [anchor=east] {clock};
  \foreach \m in {0,...,\N}
    \draw [->] (temp) -| ($ (DFF\m.CLK) + (-5mm,0) $) --(DFF\m.CLK);
%
%  % Clock port label
%  \path (DFF0) +(-2cm,-3cm) coordinate (temp)
%    node [anchor=east] {clock enable};
%  \foreach \m in {0,...,\N}
%    \draw [->] (temp) -| ($ (DFF\m.CE) + (-7.5mm,0) $) --(DFF\m.CE);
	\node at (-4,12)[block] (init) {प्रवर्ति गूढ़वाचक};    
%	\node at (4,-2)[block] (delay) {Delay};	
    \foreach \i [count=\ni from 0] in {-3,-1,1,3}
    { 
      \draw  (DFF\ni.Q) -- +({\ni+1},0) node (output\ni){\textbullet};
      \draw[->] (output\ni) |- ([yshift=\i * 0.2 cm]init.east) ;
      \draw[->] ([xshift=\i * 0.2 cm]init.south) |- (DFF\ni.D);
%       -- ([xshift=\i * 0.2 cm]identify.north) ;      


      
    }
    \foreach [count=\i] \j in {W,X,Y,Z}{
%            \node (\i) at ( 1.6, 1.2-\i*0.4) {\j} ;
%            \node (\i) at ($([yshift=\i * 0.4 cm]init.east)-(0,1)$) {\j} ;
            \node (\i) at ($([yshift=\i * 0.4 cm]init.east)-(-0.2,0.8)$) {\scriptsize \j} ;
            }
\foreach [count=\i] \j in {A,B,C,D}{
            \node (\i) at ($([xshift=\i * 0.4 cm]init.south)-(0.9,0.2)$) {\scriptsize \j} ;
            }

	
\end{tikzpicture}

%\end{document}

}
\caption{D-द्विविध द्वारा परिमित अवस्था यंत्र का कार्यान्वयन।}
\label{fig:dff}
\end{figure}
%
आकृति \ref{fig:dff} में यंत्रोपवस्तु मूल्य  सारणी \ref{table:fsm_counter} में प्रदत्त है।
\begin{table}[!h]
\resizebox {\columnwidth} {!} {
%%%%%%%%%%%%%%%%%%%%%%%%%%%%%%%%%%%%%%%%%%%%%%%%%%%%%%%%%%%%%%%%%%%%%%
%%                                                                  %%
%%  This is the header of a LaTeX2e file exported from Gnumeric.    %%
%%                                                                  %%
%%  This file can be compiled as it stands or included in another   %%
%%  LaTeX document. The table is based on the longtable package so  %%
%%  the longtable options (headers, footers...) can be set in the   %%
%%  preamble section below (see PRAMBLE).                           %%
%%                                                                  %%
%%  To include the file in another, the following two lines must be %%
%%  in the including file:                                          %%
%%        \def\inputGnumericTable{}                                 %%
%%  at the beginning of the file and:                               %%
%%        \input{name-of-this-file.tex}                             %%
%%  where the table is to be placed. Note also that the including   %%
%%  file must use the following packages for the table to be        %%
%%  rendered correctly:                                             %%
%%    \usepackage[latin1]{inputenc}                                 %%
%%    \usepackage{color}                                            %%
%%    \usepackage{array}                                            %%
%%    \usepackage{longtable}                                        %%
%%    \usepackage{calc}                                             %%
%%    \usepackage{multirow}                                         %%
%%    \usepackage{hhline}                                           %%
%%    \usepackage{ifthen}                                           %%
%%  optionally (for landscape tables embedded in another document): %%
%%    \usepackage{lscape}                                           %%
%%                                                                  %%
%%%%%%%%%%%%%%%%%%%%%%%%%%%%%%%%%%%%%%%%%%%%%%%%%%%%%%%%%%%%%%%%%%%%%%



%%  This section checks if we are begin input into another file or  %%
%%  the file will be compiled alone. First use a macro taken from   %%
%%  the TeXbook ex 7.7 (suggestion of Han-Wen Nienhuys).            %%
\def\ifundefined#1{\expandafter\ifx\csname#1\endcsname\relax}


%%  Check for the \def token for inputed files. If it is not        %%
%%  defined, the file will be processed as a standalone and the     %%
%%  preamble will be used.                                          %%
\ifundefined{inputGnumericTable}

%%  We must be able to close or not the document at the end.        %%
	\def\gnumericTableEnd{\end{document}}


%%%%%%%%%%%%%%%%%%%%%%%%%%%%%%%%%%%%%%%%%%%%%%%%%%%%%%%%%%%%%%%%%%%%%%
%%                                                                  %%
%%  This is the PREAMBLE. Change these values to get the right      %%
%%  paper size and other niceties.                                  %%
%%                                                                  %%
%%%%%%%%%%%%%%%%%%%%%%%%%%%%%%%%%%%%%%%%%%%%%%%%%%%%%%%%%%%%%%%%%%%%%%

	\documentclass[12pt%
			  %,landscape%
                    ]{report}
       \usepackage[latin1]{inputenc}
       \usepackage{fullpage}
       \usepackage{color}
       \usepackage{array}
       \usepackage{longtable}
       \usepackage{calc}
       \usepackage{multirow}
       \usepackage{hhline}
       \usepackage{ifthen}

	\begin{document}


%%  End of the preamble for the standalone. The next section is for %%
%%  documents which are included into other LaTeX2e files.          %%
\else

%%  We are not a stand alone document. For a regular table, we will %%
%%  have no preamble and only define the closing to mean nothing.   %%
    \def\gnumericTableEnd{}

%%  If we want landscape mode in an embedded document, comment out  %%
%%  the line above and uncomment the two below. The table will      %%
%%  begin on a new page and run in landscape mode.                  %%
%       \def\gnumericTableEnd{\end{landscape}}
%       \begin{landscape}


%%  End of the else clause for this file being \input.              %%
\fi

%%%%%%%%%%%%%%%%%%%%%%%%%%%%%%%%%%%%%%%%%%%%%%%%%%%%%%%%%%%%%%%%%%%%%%
%%                                                                  %%
%%  The rest is the gnumeric table, except for the closing          %%
%%  statement. Changes below will alter the table's appearance.     %%
%%                                                                  %%
%%%%%%%%%%%%%%%%%%%%%%%%%%%%%%%%%%%%%%%%%%%%%%%%%%%%%%%%%%%%%%%%%%%%%%

\providecommand{\gnumericmathit}[1]{#1} 
%%  Uncomment the next line if you would like your numbers to be in %%
%%  italics if they are italizised in the gnumeric table.           %%
%\renewcommand{\gnumericmathit}[1]{\mathit{#1}}
\providecommand{\gnumericPB}[1]%
{\let\gnumericTemp=\\#1\let\\=\gnumericTemp\hspace{0pt}}
 \ifundefined{gnumericTableWidthDefined}
        \newlength{\gnumericTableWidth}
        \newlength{\gnumericTableWidthComplete}
        \newlength{\gnumericMultiRowLength}
        \global\def\gnumericTableWidthDefined{}
 \fi
%% The following setting protects this code from babel shorthands.  %%
 \ifthenelse{\isundefined{\languageshorthands}}{}{\languageshorthands{english}}
%%  The default table format retains the relative column widths of  %%
%%  gnumeric. They can easily be changed to c, r or l. In that case %%
%%  you may want to comment out the next line and uncomment the one %%
%%  thereafter                                                      %%
\providecommand\gnumbox{\makebox[0pt]}
%%\providecommand\gnumbox[1][]{\makebox}

%% to adjust positions in multirow situations                       %%
\setlength{\bigstrutjot}{\jot}
\setlength{\extrarowheight}{\doublerulesep}

%%  The \setlongtables command keeps column widths the same across  %%
%%  pages. Simply comment out next line for varying column widths.  %%
\setlongtables

\setlength\gnumericTableWidth{%
	53pt+%
	53pt+%
	53pt+%
0pt}
\def\gumericNumCols{3}
\setlength\gnumericTableWidthComplete{\gnumericTableWidth+%
         \tabcolsep*\gumericNumCols*2+\arrayrulewidth*\gumericNumCols}
\ifthenelse{\lengthtest{\gnumericTableWidthComplete > \linewidth}}%
         {\def\gnumericScale{1*\ratio{\linewidth-%
                        \tabcolsep*\gumericNumCols*2-%
                        \arrayrulewidth*\gumericNumCols}%
{\gnumericTableWidth}}}%
{\def\gnumericScale{1}}

%%%%%%%%%%%%%%%%%%%%%%%%%%%%%%%%%%%%%%%%%%%%%%%%%%%%%%%%%%%%%%%%%%%%%%
%%                                                                  %%
%% The following are the widths of the various columns. We are      %%
%% defining them here because then they are easier to change.       %%
%% Depending on the cell formats we may use them more than once.    %%
%%                                                                  %%
%%%%%%%%%%%%%%%%%%%%%%%%%%%%%%%%%%%%%%%%%%%%%%%%%%%%%%%%%%%%%%%%%%%%%%

\ifthenelse{\isundefined{\gnumericColA}}{\newlength{\gnumericColA}}{}\settowidth{\gnumericColA}{\begin{tabular}{@{}p{53pt*\gnumericScale}@{}}x\end{tabular}}
\ifthenelse{\isundefined{\gnumericColB}}{\newlength{\gnumericColB}}{}\settowidth{\gnumericColB}{\begin{tabular}{@{}p{53pt*\gnumericScale}@{}}x\end{tabular}}
\ifthenelse{\isundefined{\gnumericColC}}{\newlength{\gnumericColC}}{}\settowidth{\gnumericColC}{\begin{tabular}{@{}p{53pt*\gnumericScale}@{}}x\end{tabular}}

\begin{tabular}[c]{%
	b{\gnumericColA}%
	b{\gnumericColB}%
	b{\gnumericColC}%
	}

%%%%%%%%%%%%%%%%%%%%%%%%%%%%%%%%%%%%%%%%%%%%%%%%%%%%%%%%%%%%%%%%%%%%%%
%%  The longtable options. (Caption, headers... see Goosens, p.124) %%
%	\caption{The Table Caption.}             \\	%
% \hline	% Across the top of the table.
%%  The rest of these options are table rows which are placed on    %%
%%  the first, last or every page. Use \multicolumn if you want.    %%

%%  Header for the first page.                                      %%
%	\multicolumn{3}{c}{The First Header} \\ \hline 
%	\multicolumn{1}{c}{colTag}	%Column 1
%	&\multicolumn{1}{c}{colTag}	%Column 2
%	&\multicolumn{1}{c}{colTag}	\\ \hline %Last column
%	\endfirsthead

%%  The running header definition.                                  %%
%	\hline
%	\multicolumn{3}{l}{\ldots\small\slshape continued} \\ \hline
%	\multicolumn{1}{c}{colTag}	%Column 1
%	&\multicolumn{1}{c}{colTag}	%Column 2
%	&\multicolumn{1}{c}{colTag}	\\ \hline %Last column
%	\endhead

%%  The running footer definition.                                  %%
%	\hline
%	\multicolumn{3}{r}{\small\slshape continued\ldots} \\
%	\endfoot

%%  The ending footer definition.                                   %%
%	\multicolumn{3}{c}{That's all folks} \\ \hline 
%	\endlastfoot
%%%%%%%%%%%%%%%%%%%%%%%%%%%%%%%%%%%%%%%%%%%%%%%%%%%%%%%%%%%%%%%%%%%%%%

\hhline{|-|-~}
	 \multicolumn{1}{|p{\gnumericColA}|}%
	{\gnumericPB{\centering}\gnumbox{\textbf{वस्त}}}
	&\multicolumn{1}{p{\gnumericColB}|}%
	{\gnumericPB{\centering}\gnumbox{\textbf{मूल्य}}}
	&
\\
\hhline{|--|~}
	 \multicolumn{1}{|p{\gnumericColA}|}%
	{\gnumericPB{\centering}\gnumbox{अवस्था ($N$)}}
	&\multicolumn{1}{p{\gnumericColB}|}%
	{\gnumericPB{\centering}\gnumbox{10}}
	&
\\
\hhline{|--|~}
	 \multicolumn{1}{|p{\gnumericColA}|}%
	{\gnumericPB{\centering}\gnumbox{द्विविध}}
	&\multicolumn{1}{p{\gnumericColB}|}%
	{\gnumericPB{\centering}\gnumbox{$\ceil{\log_2{N}} = 4$}}
	&
\\
\hhline{|-|-|~}
\end{tabular}

\ifthenelse{\isundefined{\languageshorthands}}{}{\languageshorthands{\languagename}}
\gnumericTableEnd

}
\caption{यंत्रोपवस्तु मूल्य।}
\label{table:fsm_counter}
\end{table}

उपरोक्त विधान से पूर्ववर्ती गूढ़वाचक का अभिकल्प करें।

%\begin{equation}
%\text{No. of D Flip-Flops} = \ceil{\log_{2}\brak{\text{No. of States}}}
%\end{equation}
%For the FSM in Fig. \ref{fig:fsm_counter}, the number of states is 9, hence the number flipflops required = 4.  
%\begin{problem}
%Design a decade down counter (counts from 9 to 0 repeatedly) using an FSM.  
%\end{problem}

\end{document}



}
\caption{दशक गणित्र का खंड आरेख}
\label{fig:dec_counter}
\end{figure}
%
\subsection{परिमित अवस्था यंत्र}
%
\ref{fig:dec_counter} का परिमित अवस्था यंत्र आरेख आकृति. \ref{fig:fsm_counter} में उपलब्ध है।  $s_0$ वह अवस्था है जहां परवर्ती गूढ़वाचक का आगत मूल्य  0 है।  दशक गणित्र  की अवस्थान्तरण सारणी   \ref{table:counter_decoder}  में प्रस्तुत हैं।  इसमें वर्तमान अवस्था चर  का प्रबोधन  $W,X,Y,Z$ से है एवं आगामी अवस्था चर  $A,B,C,D$ द्वारा प्रबोधित है।
\begin{figure}[!h]
\centering
\resizebox {\columnwidth} {!} {
\usetikzlibrary{arrows,automata, positioning, calc}
%\usetikzlibrary{arrows,automata, calc}
%\begin{tikzpicture}[->,shorten >=1pt,node distance=2cm,on grid,auto] 
\begin{tikzpicture}[->,auto] 
   \node[ ] (s_00)   {}; 
   \foreach \i [count=\ni from 1] in {36,72,...,324}
%       \node[state] (s_\ni) [above right = {2*sin(\i)} and {2*(cos(\i)} of s_00]  {\ni};
        \node[state] (s_\ni) [above right = {2*sin(\i)} and {2*(cos(\i)} of s_00]  {$s_{\ni}$};        
        
        \node[state,initial] (s_0) [above right = {0} and {2} of s_00]  {$s_0$};     

   \foreach \i  [count=\j from 1] in {0,1,...,8}
		\path	(s_\i) edge [bend right]  (s_\j) ;

		\path	(s_9) edge [bend right]  (s_0) ;		
           
\end{tikzpicture}

}
\caption{दशक गणित्र की अवस्थायें।}
\label{fig:fsm_counter}
\end{figure}
%
\subsection{परिमित अवस्था यंत्र के द्वारा दशक गणित्र का अभिकल्प}

 आकृति. \ref{fig:dff} में D-द्विविध के व्यूह से दशक गणित्र का अभिकल्प उपलब्ध है।  यहाँ D-द्विविध आकृति \ref{fig:dec_counter} में अतिकाल खंड को  आकृति \ref{fig:dff} के परिपथ में कार्यान्वित करता है। D-द्विविध आगत मूल्य को घड़ी के आवर्त समय के पश्चात निर्गमन
करता है।
\begin{figure}[!h]
\resizebox {\columnwidth} {!} {
%\documentclass{standalone}
%\usepackage{pgf,tikz}
%\usetikzlibrary{calc,arrows}
%\usepackage{amsmath}

\makeatletter

% Data Flip Flip (DFF) shape
\pgfdeclareshape{dff}{
  % The 'minimum width' and 'minimum height' keys, not the content, determine
  % the size
  \savedanchor\northeast{%
    \pgfmathsetlength\pgf@x{\pgfshapeminwidth}%
    \pgfmathsetlength\pgf@y{\pgfshapeminheight}%
    \pgf@x=0.5\pgf@x
    \pgf@y=0.5\pgf@y
  }
  % This is redundant, but makes some things easier:
  \savedanchor\southwest{%
    \pgfmathsetlength\pgf@x{\pgfshapeminwidth}%
    \pgfmathsetlength\pgf@y{\pgfshapeminheight}%
    \pgf@x=-0.5\pgf@x
    \pgf@y=-0.5\pgf@y
  }
  % Inherit from rectangle
  \inheritanchorborder[from=rectangle]

  % Define same anchor a normal rectangle has
  \anchor{center}{\pgfpointorigin}
  \anchor{north}{\northeast \pgf@x=0pt}
  \anchor{east}{\northeast \pgf@y=0pt}
  \anchor{south}{\southwest \pgf@x=0pt}
  \anchor{west}{\southwest \pgf@y=0pt}
  \anchor{north east}{\northeast}
  \anchor{north west}{\northeast \pgf@x=-\pgf@x}
  \anchor{south west}{\southwest}
  \anchor{south east}{\southwest \pgf@x=-\pgf@x}
  \anchor{text}{
    \pgfpointorigin
    \advance\pgf@x by -.5\wd\pgfnodeparttextbox%
    \advance\pgf@y by -.5\ht\pgfnodeparttextbox%
    \advance\pgf@y by +.5\dp\pgfnodeparttextbox%
  }

  % Define anchors for signal ports
  \anchor{D}{
    \pgf@process{\northeast}%
    \pgf@x=-1\pgf@x%
    \pgf@y=.5\pgf@y%
  }
  \anchor{CLK}{
    \pgf@process{\northeast}%
    \pgf@x=-1\pgf@x%
    \pgf@y=-.66666\pgf@y%
  }
  \anchor{CE}{
    \pgf@process{\northeast}%
    \pgf@x=-1\pgf@x%
    \pgf@y=-0.33333\pgf@y%
  }
  \anchor{Q}{
    \pgf@process{\northeast}%
    \pgf@y=.5\pgf@y%
  }
  \anchor{Qn}{
    \pgf@process{\northeast}%
    \pgf@y=-.5\pgf@y%
  }
  \anchor{R}{
    \pgf@process{\northeast}%
    \pgf@x=0pt%
  }
  \anchor{S}{
    \pgf@process{\northeast}%
    \pgf@x=0pt%
    \pgf@y=-\pgf@y%
  }
  % Draw the rectangle box and the port labels
  \backgroundpath{
    % Rectangle box
    \pgfpathrectanglecorners{\southwest}{\northeast}
    % Angle (>) for clock input
    \pgf@anchor@dff@CLK
    \pgf@xa=\pgf@x \pgf@ya=\pgf@y
    \pgf@xb=\pgf@x \pgf@yb=\pgf@y
    \pgf@xc=\pgf@x \pgf@yc=\pgf@y
    \pgfmathsetlength\pgf@x{1.6ex} % size depends on font size
    \advance\pgf@ya by \pgf@x
    \advance\pgf@xb by \pgf@x
    \advance\pgf@yc by -\pgf@x
    \pgfpathmoveto{\pgfpoint{\pgf@xa}{\pgf@ya}}
    \pgfpathlineto{\pgfpoint{\pgf@xb}{\pgf@yb}}
    \pgfpathlineto{\pgfpoint{\pgf@xc}{\pgf@yc}}
    \pgfclosepath

    % Draw port labels
    \begingroup
    \tikzset{flip flop/port labels} % Use font from this style
    \tikz@textfont

    \pgf@anchor@dff@D
    \pgftext[left,base,at={\pgfpoint{\pgf@x}{\pgf@y}},x=\pgfshapeinnerxsep]{\raisebox{-0.75ex}{D}}

    \pgf@anchor@dff@CE
    \pgftext[left,base,at={\pgfpoint{\pgf@x}{\pgf@y}},x=\pgfshapeinnerxsep]{\raisebox{-0.75ex}{CE}}

    \pgf@anchor@dff@Q
    \pgftext[right,base,at={\pgfpoint{\pgf@x}{\pgf@y}},x=-\pgfshapeinnerxsep]{\raisebox{-.75ex}{Q}}

    \pgf@anchor@dff@Qn
    \pgftext[right,base,at={\pgfpoint{\pgf@x}{\pgf@y}},x=-\pgfshapeinnerxsep]{\raisebox{-.75ex}{$\overline{\mbox{Q}}$}}

    \pgf@anchor@dff@R
    \pgftext[top,at={\pgfpoint{\pgf@x}{\pgf@y}},y=-\pgfshapeinnerysep]{R}

    \pgf@anchor@dff@S
    \pgftext[bottom,at={\pgfpoint{\pgf@x}{\pgf@y}},y=\pgfshapeinnerysep]{S}
    \endgroup
  }
}

% Key to add font macros to the current font
\tikzset{add font/.code={\expandafter\def\expandafter\tikz@textfont\expandafter{\tikz@textfont#1}}} 

% Define default style for this node
\tikzset{flip flop/port labels/.style={font=\sffamily\scriptsize}}
\tikzset{every dff node/.style={draw,minimum width=2cm,minimum 
height=2.828427125cm,very thick,inner sep=1mm,outer sep=0pt,cap=round,add 
font=\sffamily}}
\tikzstyle{block} = [rectangle, draw, 
    text width=5em, text centered, rounded corners, minimum height=4em]


%\makeatother

%\begin{document}

\begin{tikzpicture}[font=\sffamily,>=triangle 45]
  \def\N{3}  % Number of Flip-Flops minus one

  % Place FFs
    \foreach \i [count=\m from 0] in {A,B,C,D}  
       \node [shape=dff] (DFF\m) at ($ 3*(0,\m) $) {\i};
%  \foreach \m in {0,...,\N}
%    \node [shape=dff] (DFF\m) at ($ 3*(0,\m) $) {Bit \#\m};

%  \def\N{7}  % Number of Flip-Flops minus one
%
%  % Place FFs
%  \foreach \m in {0,...,\N}
%    \node [shape=dff] (DFF\m) at ($ 3*(\m,0) $) {Bit \#\m};
%
%  % Connect FFs (Q1 with D1, etc.)
%  \def\p{0}  % Used to save the previous number
%  \foreach \m in {1,...,\N} { % Note that it starts with 1, not 0
%    \draw [->] (DFF\p.Q) -- (DFF\m.D);
%    \global\let\p\m
%  }
%
  % Connect and label data in- and output port
%  \draw [<-] (DFF0.D) -- +(-1,0) node [anchor=east] {input} ;
%  \draw [->] (DFF\N.Q) -- +(1,0) node [anchor=west] {output};
%
%  % 'Reset' port label
%  \path (DFF0) +(-2cm,+2cm) coordinate (temp)
%    node [anchor=east] {reset};
%  % Connect resets
%  \foreach \m in {0,...,\N}
%    \draw [->] (temp) -| (DFF\m.R);
%
%  % 'Set' port label
%  \path (DFF0) +(-2cm,-2cm) coordinate (temp)
%    node [anchor=east] {set};
%  % Connect sets
%  \foreach \m in {0,...,\N}
%    \draw [->] (temp) -| (DFF\m.S);
%
  % Clock port label
  \path (DFF0) +(-2cm,-2.5cm) coordinate (temp)
    node [anchor=east] {clock};
  \foreach \m in {0,...,\N}
    \draw [->] (temp) -| ($ (DFF\m.CLK) + (-5mm,0) $) --(DFF\m.CLK);
%
%  % Clock port label
%  \path (DFF0) +(-2cm,-3cm) coordinate (temp)
%    node [anchor=east] {clock enable};
%  \foreach \m in {0,...,\N}
%    \draw [->] (temp) -| ($ (DFF\m.CE) + (-7.5mm,0) $) --(DFF\m.CE);
	\node at (-4,12)[block] (init) {प्रवर्ति गूढ़वाचक};    
%	\node at (4,-2)[block] (delay) {Delay};	
    \foreach \i [count=\ni from 0] in {-3,-1,1,3}
    { 
      \draw  (DFF\ni.Q) -- +({\ni+1},0) node (output\ni){\textbullet};
      \draw[->] (output\ni) |- ([yshift=\i * 0.2 cm]init.east) ;
      \draw[->] ([xshift=\i * 0.2 cm]init.south) |- (DFF\ni.D);
%       -- ([xshift=\i * 0.2 cm]identify.north) ;      


      
    }
    \foreach [count=\i] \j in {W,X,Y,Z}{
%            \node (\i) at ( 1.6, 1.2-\i*0.4) {\j} ;
%            \node (\i) at ($([yshift=\i * 0.4 cm]init.east)-(0,1)$) {\j} ;
            \node (\i) at ($([yshift=\i * 0.4 cm]init.east)-(-0.2,0.8)$) {\scriptsize \j} ;
            }
\foreach [count=\i] \j in {A,B,C,D}{
            \node (\i) at ($([xshift=\i * 0.4 cm]init.south)-(0.9,0.2)$) {\scriptsize \j} ;
            }

	
\end{tikzpicture}

%\end{document}

}
\caption{D-द्विविध द्वारा परिमित अवस्था यंत्र का कार्यान्वयन।}
\label{fig:dff}
\end{figure}
%
आकृति \ref{fig:dff} में यंत्रोपवस्तु मूल्य  सारणी \ref{table:fsm_counter} में प्रदत्त है।
\begin{table}[!h]
\resizebox {\columnwidth} {!} {
%%%%%%%%%%%%%%%%%%%%%%%%%%%%%%%%%%%%%%%%%%%%%%%%%%%%%%%%%%%%%%%%%%%%%%
%%                                                                  %%
%%  This is the header of a LaTeX2e file exported from Gnumeric.    %%
%%                                                                  %%
%%  This file can be compiled as it stands or included in another   %%
%%  LaTeX document. The table is based on the longtable package so  %%
%%  the longtable options (headers, footers...) can be set in the   %%
%%  preamble section below (see PRAMBLE).                           %%
%%                                                                  %%
%%  To include the file in another, the following two lines must be %%
%%  in the including file:                                          %%
%%        \def\inputGnumericTable{}                                 %%
%%  at the beginning of the file and:                               %%
%%        \input{name-of-this-file.tex}                             %%
%%  where the table is to be placed. Note also that the including   %%
%%  file must use the following packages for the table to be        %%
%%  rendered correctly:                                             %%
%%    \usepackage[latin1]{inputenc}                                 %%
%%    \usepackage{color}                                            %%
%%    \usepackage{array}                                            %%
%%    \usepackage{longtable}                                        %%
%%    \usepackage{calc}                                             %%
%%    \usepackage{multirow}                                         %%
%%    \usepackage{hhline}                                           %%
%%    \usepackage{ifthen}                                           %%
%%  optionally (for landscape tables embedded in another document): %%
%%    \usepackage{lscape}                                           %%
%%                                                                  %%
%%%%%%%%%%%%%%%%%%%%%%%%%%%%%%%%%%%%%%%%%%%%%%%%%%%%%%%%%%%%%%%%%%%%%%



%%  This section checks if we are begin input into another file or  %%
%%  the file will be compiled alone. First use a macro taken from   %%
%%  the TeXbook ex 7.7 (suggestion of Han-Wen Nienhuys).            %%
\def\ifundefined#1{\expandafter\ifx\csname#1\endcsname\relax}


%%  Check for the \def token for inputed files. If it is not        %%
%%  defined, the file will be processed as a standalone and the     %%
%%  preamble will be used.                                          %%
\ifundefined{inputGnumericTable}

%%  We must be able to close or not the document at the end.        %%
	\def\gnumericTableEnd{\end{document}}


%%%%%%%%%%%%%%%%%%%%%%%%%%%%%%%%%%%%%%%%%%%%%%%%%%%%%%%%%%%%%%%%%%%%%%
%%                                                                  %%
%%  This is the PREAMBLE. Change these values to get the right      %%
%%  paper size and other niceties.                                  %%
%%                                                                  %%
%%%%%%%%%%%%%%%%%%%%%%%%%%%%%%%%%%%%%%%%%%%%%%%%%%%%%%%%%%%%%%%%%%%%%%

	\documentclass[12pt%
			  %,landscape%
                    ]{report}
       \usepackage[latin1]{inputenc}
       \usepackage{fullpage}
       \usepackage{color}
       \usepackage{array}
       \usepackage{longtable}
       \usepackage{calc}
       \usepackage{multirow}
       \usepackage{hhline}
       \usepackage{ifthen}

	\begin{document}


%%  End of the preamble for the standalone. The next section is for %%
%%  documents which are included into other LaTeX2e files.          %%
\else

%%  We are not a stand alone document. For a regular table, we will %%
%%  have no preamble and only define the closing to mean nothing.   %%
    \def\gnumericTableEnd{}

%%  If we want landscape mode in an embedded document, comment out  %%
%%  the line above and uncomment the two below. The table will      %%
%%  begin on a new page and run in landscape mode.                  %%
%       \def\gnumericTableEnd{\end{landscape}}
%       \begin{landscape}


%%  End of the else clause for this file being \input.              %%
\fi

%%%%%%%%%%%%%%%%%%%%%%%%%%%%%%%%%%%%%%%%%%%%%%%%%%%%%%%%%%%%%%%%%%%%%%
%%                                                                  %%
%%  The rest is the gnumeric table, except for the closing          %%
%%  statement. Changes below will alter the table's appearance.     %%
%%                                                                  %%
%%%%%%%%%%%%%%%%%%%%%%%%%%%%%%%%%%%%%%%%%%%%%%%%%%%%%%%%%%%%%%%%%%%%%%

\providecommand{\gnumericmathit}[1]{#1} 
%%  Uncomment the next line if you would like your numbers to be in %%
%%  italics if they are italizised in the gnumeric table.           %%
%\renewcommand{\gnumericmathit}[1]{\mathit{#1}}
\providecommand{\gnumericPB}[1]%
{\let\gnumericTemp=\\#1\let\\=\gnumericTemp\hspace{0pt}}
 \ifundefined{gnumericTableWidthDefined}
        \newlength{\gnumericTableWidth}
        \newlength{\gnumericTableWidthComplete}
        \newlength{\gnumericMultiRowLength}
        \global\def\gnumericTableWidthDefined{}
 \fi
%% The following setting protects this code from babel shorthands.  %%
 \ifthenelse{\isundefined{\languageshorthands}}{}{\languageshorthands{english}}
%%  The default table format retains the relative column widths of  %%
%%  gnumeric. They can easily be changed to c, r or l. In that case %%
%%  you may want to comment out the next line and uncomment the one %%
%%  thereafter                                                      %%
\providecommand\gnumbox{\makebox[0pt]}
%%\providecommand\gnumbox[1][]{\makebox}

%% to adjust positions in multirow situations                       %%
\setlength{\bigstrutjot}{\jot}
\setlength{\extrarowheight}{\doublerulesep}

%%  The \setlongtables command keeps column widths the same across  %%
%%  pages. Simply comment out next line for varying column widths.  %%
\setlongtables

\setlength\gnumericTableWidth{%
	53pt+%
	53pt+%
	53pt+%
0pt}
\def\gumericNumCols{3}
\setlength\gnumericTableWidthComplete{\gnumericTableWidth+%
         \tabcolsep*\gumericNumCols*2+\arrayrulewidth*\gumericNumCols}
\ifthenelse{\lengthtest{\gnumericTableWidthComplete > \linewidth}}%
         {\def\gnumericScale{1*\ratio{\linewidth-%
                        \tabcolsep*\gumericNumCols*2-%
                        \arrayrulewidth*\gumericNumCols}%
{\gnumericTableWidth}}}%
{\def\gnumericScale{1}}

%%%%%%%%%%%%%%%%%%%%%%%%%%%%%%%%%%%%%%%%%%%%%%%%%%%%%%%%%%%%%%%%%%%%%%
%%                                                                  %%
%% The following are the widths of the various columns. We are      %%
%% defining them here because then they are easier to change.       %%
%% Depending on the cell formats we may use them more than once.    %%
%%                                                                  %%
%%%%%%%%%%%%%%%%%%%%%%%%%%%%%%%%%%%%%%%%%%%%%%%%%%%%%%%%%%%%%%%%%%%%%%

\ifthenelse{\isundefined{\gnumericColA}}{\newlength{\gnumericColA}}{}\settowidth{\gnumericColA}{\begin{tabular}{@{}p{53pt*\gnumericScale}@{}}x\end{tabular}}
\ifthenelse{\isundefined{\gnumericColB}}{\newlength{\gnumericColB}}{}\settowidth{\gnumericColB}{\begin{tabular}{@{}p{53pt*\gnumericScale}@{}}x\end{tabular}}
\ifthenelse{\isundefined{\gnumericColC}}{\newlength{\gnumericColC}}{}\settowidth{\gnumericColC}{\begin{tabular}{@{}p{53pt*\gnumericScale}@{}}x\end{tabular}}

\begin{tabular}[c]{%
	b{\gnumericColA}%
	b{\gnumericColB}%
	b{\gnumericColC}%
	}

%%%%%%%%%%%%%%%%%%%%%%%%%%%%%%%%%%%%%%%%%%%%%%%%%%%%%%%%%%%%%%%%%%%%%%
%%  The longtable options. (Caption, headers... see Goosens, p.124) %%
%	\caption{The Table Caption.}             \\	%
% \hline	% Across the top of the table.
%%  The rest of these options are table rows which are placed on    %%
%%  the first, last or every page. Use \multicolumn if you want.    %%

%%  Header for the first page.                                      %%
%	\multicolumn{3}{c}{The First Header} \\ \hline 
%	\multicolumn{1}{c}{colTag}	%Column 1
%	&\multicolumn{1}{c}{colTag}	%Column 2
%	&\multicolumn{1}{c}{colTag}	\\ \hline %Last column
%	\endfirsthead

%%  The running header definition.                                  %%
%	\hline
%	\multicolumn{3}{l}{\ldots\small\slshape continued} \\ \hline
%	\multicolumn{1}{c}{colTag}	%Column 1
%	&\multicolumn{1}{c}{colTag}	%Column 2
%	&\multicolumn{1}{c}{colTag}	\\ \hline %Last column
%	\endhead

%%  The running footer definition.                                  %%
%	\hline
%	\multicolumn{3}{r}{\small\slshape continued\ldots} \\
%	\endfoot

%%  The ending footer definition.                                   %%
%	\multicolumn{3}{c}{That's all folks} \\ \hline 
%	\endlastfoot
%%%%%%%%%%%%%%%%%%%%%%%%%%%%%%%%%%%%%%%%%%%%%%%%%%%%%%%%%%%%%%%%%%%%%%

\hhline{|-|-~}
	 \multicolumn{1}{|p{\gnumericColA}|}%
	{\gnumericPB{\centering}\gnumbox{\textbf{वस्त}}}
	&\multicolumn{1}{p{\gnumericColB}|}%
	{\gnumericPB{\centering}\gnumbox{\textbf{मूल्य}}}
	&
\\
\hhline{|--|~}
	 \multicolumn{1}{|p{\gnumericColA}|}%
	{\gnumericPB{\centering}\gnumbox{अवस्था ($N$)}}
	&\multicolumn{1}{p{\gnumericColB}|}%
	{\gnumericPB{\centering}\gnumbox{10}}
	&
\\
\hhline{|--|~}
	 \multicolumn{1}{|p{\gnumericColA}|}%
	{\gnumericPB{\centering}\gnumbox{द्विविध}}
	&\multicolumn{1}{p{\gnumericColB}|}%
	{\gnumericPB{\centering}\gnumbox{$\ceil{\log_2{N}} = 4$}}
	&
\\
\hhline{|-|-|~}
\end{tabular}

\ifthenelse{\isundefined{\languageshorthands}}{}{\languageshorthands{\languagename}}
\gnumericTableEnd

}
\caption{यंत्रोपवस्तु मूल्य।}
\label{table:fsm_counter}
\end{table}

उपरोक्त विधान से पूर्ववर्ती गूढ़वाचक का अभिकल्प करें।

%\begin{equation}
%\text{No. of D Flip-Flops} = \ceil{\log_{2}\brak{\text{No. of States}}}
%\end{equation}
%For the FSM in Fig. \ref{fig:fsm_counter}, the number of states is 9, hence the number flipflops required = 4.  
%\begin{problem}
%Design a decade down counter (counts from 9 to 0 repeatedly) using an FSM.  
%\end{problem}

\end{document}



}
\caption{दशक गणित्र का खंड आरेख}
\label{fig:dec_counter}
\end{figure}
%
\subsection{परिमित अवस्था यंत्र}
%
\ref{fig:dec_counter} का परिमित अवस्था यंत्र आरेख आकृति. \ref{fig:fsm_counter} में उपलब्ध है।  $s_0$ वह अवस्था है जहां परवर्ती गूढ़वाचक का आगत मूल्य  0 है।  दशक गणित्र  की अवस्थान्तरण सारणी   \ref{table:counter_decoder}  में प्रस्तुत हैं।  इसमें वर्तमान अवस्था चर  का प्रबोधन  $W,X,Y,Z$ से है एवं आगामी अवस्था चर  $A,B,C,D$ द्वारा प्रबोधित है।
\begin{figure}[!h]
\centering
\resizebox {\columnwidth} {!} {
\usetikzlibrary{arrows,automata, positioning, calc}
%\usetikzlibrary{arrows,automata, calc}
%\begin{tikzpicture}[->,shorten >=1pt,node distance=2cm,on grid,auto] 
\begin{tikzpicture}[->,auto] 
   \node[ ] (s_00)   {}; 
   \foreach \i [count=\ni from 1] in {36,72,...,324}
%       \node[state] (s_\ni) [above right = {2*sin(\i)} and {2*(cos(\i)} of s_00]  {\ni};
        \node[state] (s_\ni) [above right = {2*sin(\i)} and {2*(cos(\i)} of s_00]  {$s_{\ni}$};        
        
        \node[state,initial] (s_0) [above right = {0} and {2} of s_00]  {$s_0$};     

   \foreach \i  [count=\j from 1] in {0,1,...,8}
		\path	(s_\i) edge [bend right]  (s_\j) ;

		\path	(s_9) edge [bend right]  (s_0) ;		
           
\end{tikzpicture}

}
\caption{दशक गणित्र की अवस्थायें।}
\label{fig:fsm_counter}
\end{figure}
%
\subsection{परिमित अवस्था यंत्र के द्वारा दशक गणित्र का अभिकल्प}

 आकृति. \ref{fig:dff} में D-द्विविध के व्यूह से दशक गणित्र का अभिकल्प उपलब्ध है।  यहाँ D-द्विविध आकृति \ref{fig:dec_counter} में अतिकाल खंड को  आकृति \ref{fig:dff} के परिपथ में कार्यान्वित करता है। D-द्विविध आगत मूल्य को घड़ी के आवर्त समय के पश्चात निर्गमन
करता है।
\begin{figure}[!h]
\resizebox {\columnwidth} {!} {
%\documentclass{standalone}
%\usepackage{pgf,tikz}
%\usetikzlibrary{calc,arrows}
%\usepackage{amsmath}

\makeatletter

% Data Flip Flip (DFF) shape
\pgfdeclareshape{dff}{
  % The 'minimum width' and 'minimum height' keys, not the content, determine
  % the size
  \savedanchor\northeast{%
    \pgfmathsetlength\pgf@x{\pgfshapeminwidth}%
    \pgfmathsetlength\pgf@y{\pgfshapeminheight}%
    \pgf@x=0.5\pgf@x
    \pgf@y=0.5\pgf@y
  }
  % This is redundant, but makes some things easier:
  \savedanchor\southwest{%
    \pgfmathsetlength\pgf@x{\pgfshapeminwidth}%
    \pgfmathsetlength\pgf@y{\pgfshapeminheight}%
    \pgf@x=-0.5\pgf@x
    \pgf@y=-0.5\pgf@y
  }
  % Inherit from rectangle
  \inheritanchorborder[from=rectangle]

  % Define same anchor a normal rectangle has
  \anchor{center}{\pgfpointorigin}
  \anchor{north}{\northeast \pgf@x=0pt}
  \anchor{east}{\northeast \pgf@y=0pt}
  \anchor{south}{\southwest \pgf@x=0pt}
  \anchor{west}{\southwest \pgf@y=0pt}
  \anchor{north east}{\northeast}
  \anchor{north west}{\northeast \pgf@x=-\pgf@x}
  \anchor{south west}{\southwest}
  \anchor{south east}{\southwest \pgf@x=-\pgf@x}
  \anchor{text}{
    \pgfpointorigin
    \advance\pgf@x by -.5\wd\pgfnodeparttextbox%
    \advance\pgf@y by -.5\ht\pgfnodeparttextbox%
    \advance\pgf@y by +.5\dp\pgfnodeparttextbox%
  }

  % Define anchors for signal ports
  \anchor{D}{
    \pgf@process{\northeast}%
    \pgf@x=-1\pgf@x%
    \pgf@y=.5\pgf@y%
  }
  \anchor{CLK}{
    \pgf@process{\northeast}%
    \pgf@x=-1\pgf@x%
    \pgf@y=-.66666\pgf@y%
  }
  \anchor{CE}{
    \pgf@process{\northeast}%
    \pgf@x=-1\pgf@x%
    \pgf@y=-0.33333\pgf@y%
  }
  \anchor{Q}{
    \pgf@process{\northeast}%
    \pgf@y=.5\pgf@y%
  }
  \anchor{Qn}{
    \pgf@process{\northeast}%
    \pgf@y=-.5\pgf@y%
  }
  \anchor{R}{
    \pgf@process{\northeast}%
    \pgf@x=0pt%
  }
  \anchor{S}{
    \pgf@process{\northeast}%
    \pgf@x=0pt%
    \pgf@y=-\pgf@y%
  }
  % Draw the rectangle box and the port labels
  \backgroundpath{
    % Rectangle box
    \pgfpathrectanglecorners{\southwest}{\northeast}
    % Angle (>) for clock input
    \pgf@anchor@dff@CLK
    \pgf@xa=\pgf@x \pgf@ya=\pgf@y
    \pgf@xb=\pgf@x \pgf@yb=\pgf@y
    \pgf@xc=\pgf@x \pgf@yc=\pgf@y
    \pgfmathsetlength\pgf@x{1.6ex} % size depends on font size
    \advance\pgf@ya by \pgf@x
    \advance\pgf@xb by \pgf@x
    \advance\pgf@yc by -\pgf@x
    \pgfpathmoveto{\pgfpoint{\pgf@xa}{\pgf@ya}}
    \pgfpathlineto{\pgfpoint{\pgf@xb}{\pgf@yb}}
    \pgfpathlineto{\pgfpoint{\pgf@xc}{\pgf@yc}}
    \pgfclosepath

    % Draw port labels
    \begingroup
    \tikzset{flip flop/port labels} % Use font from this style
    \tikz@textfont

    \pgf@anchor@dff@D
    \pgftext[left,base,at={\pgfpoint{\pgf@x}{\pgf@y}},x=\pgfshapeinnerxsep]{\raisebox{-0.75ex}{D}}

    \pgf@anchor@dff@CE
    \pgftext[left,base,at={\pgfpoint{\pgf@x}{\pgf@y}},x=\pgfshapeinnerxsep]{\raisebox{-0.75ex}{CE}}

    \pgf@anchor@dff@Q
    \pgftext[right,base,at={\pgfpoint{\pgf@x}{\pgf@y}},x=-\pgfshapeinnerxsep]{\raisebox{-.75ex}{Q}}

    \pgf@anchor@dff@Qn
    \pgftext[right,base,at={\pgfpoint{\pgf@x}{\pgf@y}},x=-\pgfshapeinnerxsep]{\raisebox{-.75ex}{$\overline{\mbox{Q}}$}}

    \pgf@anchor@dff@R
    \pgftext[top,at={\pgfpoint{\pgf@x}{\pgf@y}},y=-\pgfshapeinnerysep]{R}

    \pgf@anchor@dff@S
    \pgftext[bottom,at={\pgfpoint{\pgf@x}{\pgf@y}},y=\pgfshapeinnerysep]{S}
    \endgroup
  }
}

% Key to add font macros to the current font
\tikzset{add font/.code={\expandafter\def\expandafter\tikz@textfont\expandafter{\tikz@textfont#1}}} 

% Define default style for this node
\tikzset{flip flop/port labels/.style={font=\sffamily\scriptsize}}
\tikzset{every dff node/.style={draw,minimum width=2cm,minimum 
height=2.828427125cm,very thick,inner sep=1mm,outer sep=0pt,cap=round,add 
font=\sffamily}}
\tikzstyle{block} = [rectangle, draw, 
    text width=5em, text centered, rounded corners, minimum height=4em]


%\makeatother

%\begin{document}

\begin{tikzpicture}[font=\sffamily,>=triangle 45]
  \def\N{3}  % Number of Flip-Flops minus one

  % Place FFs
    \foreach \i [count=\m from 0] in {A,B,C,D}  
       \node [shape=dff] (DFF\m) at ($ 3*(0,\m) $) {\i};
%  \foreach \m in {0,...,\N}
%    \node [shape=dff] (DFF\m) at ($ 3*(0,\m) $) {Bit \#\m};

%  \def\N{7}  % Number of Flip-Flops minus one
%
%  % Place FFs
%  \foreach \m in {0,...,\N}
%    \node [shape=dff] (DFF\m) at ($ 3*(\m,0) $) {Bit \#\m};
%
%  % Connect FFs (Q1 with D1, etc.)
%  \def\p{0}  % Used to save the previous number
%  \foreach \m in {1,...,\N} { % Note that it starts with 1, not 0
%    \draw [->] (DFF\p.Q) -- (DFF\m.D);
%    \global\let\p\m
%  }
%
  % Connect and label data in- and output port
%  \draw [<-] (DFF0.D) -- +(-1,0) node [anchor=east] {input} ;
%  \draw [->] (DFF\N.Q) -- +(1,0) node [anchor=west] {output};
%
%  % 'Reset' port label
%  \path (DFF0) +(-2cm,+2cm) coordinate (temp)
%    node [anchor=east] {reset};
%  % Connect resets
%  \foreach \m in {0,...,\N}
%    \draw [->] (temp) -| (DFF\m.R);
%
%  % 'Set' port label
%  \path (DFF0) +(-2cm,-2cm) coordinate (temp)
%    node [anchor=east] {set};
%  % Connect sets
%  \foreach \m in {0,...,\N}
%    \draw [->] (temp) -| (DFF\m.S);
%
  % Clock port label
  \path (DFF0) +(-2cm,-2.5cm) coordinate (temp)
    node [anchor=east] {clock};
  \foreach \m in {0,...,\N}
    \draw [->] (temp) -| ($ (DFF\m.CLK) + (-5mm,0) $) --(DFF\m.CLK);
%
%  % Clock port label
%  \path (DFF0) +(-2cm,-3cm) coordinate (temp)
%    node [anchor=east] {clock enable};
%  \foreach \m in {0,...,\N}
%    \draw [->] (temp) -| ($ (DFF\m.CE) + (-7.5mm,0) $) --(DFF\m.CE);
	\node at (-4,12)[block] (init) {प्रवर्ति गूढ़वाचक};    
%	\node at (4,-2)[block] (delay) {Delay};	
    \foreach \i [count=\ni from 0] in {-3,-1,1,3}
    { 
      \draw  (DFF\ni.Q) -- +({\ni+1},0) node (output\ni){\textbullet};
      \draw[->] (output\ni) |- ([yshift=\i * 0.2 cm]init.east) ;
      \draw[->] ([xshift=\i * 0.2 cm]init.south) |- (DFF\ni.D);
%       -- ([xshift=\i * 0.2 cm]identify.north) ;      


      
    }
    \foreach [count=\i] \j in {W,X,Y,Z}{
%            \node (\i) at ( 1.6, 1.2-\i*0.4) {\j} ;
%            \node (\i) at ($([yshift=\i * 0.4 cm]init.east)-(0,1)$) {\j} ;
            \node (\i) at ($([yshift=\i * 0.4 cm]init.east)-(-0.2,0.8)$) {\scriptsize \j} ;
            }
\foreach [count=\i] \j in {A,B,C,D}{
            \node (\i) at ($([xshift=\i * 0.4 cm]init.south)-(0.9,0.2)$) {\scriptsize \j} ;
            }

	
\end{tikzpicture}

%\end{document}

}
\caption{D-द्विविध द्वारा परिमित अवस्था यंत्र का कार्यान्वयन।}
\label{fig:dff}
\end{figure}
%
आकृति \ref{fig:dff} में यंत्रोपवस्तु मूल्य  सारणी \ref{table:fsm_counter} में प्रदत्त है।
\begin{table}[!h]
\resizebox {\columnwidth} {!} {
%%%%%%%%%%%%%%%%%%%%%%%%%%%%%%%%%%%%%%%%%%%%%%%%%%%%%%%%%%%%%%%%%%%%%%
%%                                                                  %%
%%  This is the header of a LaTeX2e file exported from Gnumeric.    %%
%%                                                                  %%
%%  This file can be compiled as it stands or included in another   %%
%%  LaTeX document. The table is based on the longtable package so  %%
%%  the longtable options (headers, footers...) can be set in the   %%
%%  preamble section below (see PRAMBLE).                           %%
%%                                                                  %%
%%  To include the file in another, the following two lines must be %%
%%  in the including file:                                          %%
%%        \def\inputGnumericTable{}                                 %%
%%  at the beginning of the file and:                               %%
%%        \input{name-of-this-file.tex}                             %%
%%  where the table is to be placed. Note also that the including   %%
%%  file must use the following packages for the table to be        %%
%%  rendered correctly:                                             %%
%%    \usepackage[latin1]{inputenc}                                 %%
%%    \usepackage{color}                                            %%
%%    \usepackage{array}                                            %%
%%    \usepackage{longtable}                                        %%
%%    \usepackage{calc}                                             %%
%%    \usepackage{multirow}                                         %%
%%    \usepackage{hhline}                                           %%
%%    \usepackage{ifthen}                                           %%
%%  optionally (for landscape tables embedded in another document): %%
%%    \usepackage{lscape}                                           %%
%%                                                                  %%
%%%%%%%%%%%%%%%%%%%%%%%%%%%%%%%%%%%%%%%%%%%%%%%%%%%%%%%%%%%%%%%%%%%%%%



%%  This section checks if we are begin input into another file or  %%
%%  the file will be compiled alone. First use a macro taken from   %%
%%  the TeXbook ex 7.7 (suggestion of Han-Wen Nienhuys).            %%
\def\ifundefined#1{\expandafter\ifx\csname#1\endcsname\relax}


%%  Check for the \def token for inputed files. If it is not        %%
%%  defined, the file will be processed as a standalone and the     %%
%%  preamble will be used.                                          %%
\ifundefined{inputGnumericTable}

%%  We must be able to close or not the document at the end.        %%
	\def\gnumericTableEnd{\end{document}}


%%%%%%%%%%%%%%%%%%%%%%%%%%%%%%%%%%%%%%%%%%%%%%%%%%%%%%%%%%%%%%%%%%%%%%
%%                                                                  %%
%%  This is the PREAMBLE. Change these values to get the right      %%
%%  paper size and other niceties.                                  %%
%%                                                                  %%
%%%%%%%%%%%%%%%%%%%%%%%%%%%%%%%%%%%%%%%%%%%%%%%%%%%%%%%%%%%%%%%%%%%%%%

	\documentclass[12pt%
			  %,landscape%
                    ]{report}
       \usepackage[latin1]{inputenc}
       \usepackage{fullpage}
       \usepackage{color}
       \usepackage{array}
       \usepackage{longtable}
       \usepackage{calc}
       \usepackage{multirow}
       \usepackage{hhline}
       \usepackage{ifthen}

	\begin{document}


%%  End of the preamble for the standalone. The next section is for %%
%%  documents which are included into other LaTeX2e files.          %%
\else

%%  We are not a stand alone document. For a regular table, we will %%
%%  have no preamble and only define the closing to mean nothing.   %%
    \def\gnumericTableEnd{}

%%  If we want landscape mode in an embedded document, comment out  %%
%%  the line above and uncomment the two below. The table will      %%
%%  begin on a new page and run in landscape mode.                  %%
%       \def\gnumericTableEnd{\end{landscape}}
%       \begin{landscape}


%%  End of the else clause for this file being \input.              %%
\fi

%%%%%%%%%%%%%%%%%%%%%%%%%%%%%%%%%%%%%%%%%%%%%%%%%%%%%%%%%%%%%%%%%%%%%%
%%                                                                  %%
%%  The rest is the gnumeric table, except for the closing          %%
%%  statement. Changes below will alter the table's appearance.     %%
%%                                                                  %%
%%%%%%%%%%%%%%%%%%%%%%%%%%%%%%%%%%%%%%%%%%%%%%%%%%%%%%%%%%%%%%%%%%%%%%

\providecommand{\gnumericmathit}[1]{#1} 
%%  Uncomment the next line if you would like your numbers to be in %%
%%  italics if they are italizised in the gnumeric table.           %%
%\renewcommand{\gnumericmathit}[1]{\mathit{#1}}
\providecommand{\gnumericPB}[1]%
{\let\gnumericTemp=\\#1\let\\=\gnumericTemp\hspace{0pt}}
 \ifundefined{gnumericTableWidthDefined}
        \newlength{\gnumericTableWidth}
        \newlength{\gnumericTableWidthComplete}
        \newlength{\gnumericMultiRowLength}
        \global\def\gnumericTableWidthDefined{}
 \fi
%% The following setting protects this code from babel shorthands.  %%
 \ifthenelse{\isundefined{\languageshorthands}}{}{\languageshorthands{english}}
%%  The default table format retains the relative column widths of  %%
%%  gnumeric. They can easily be changed to c, r or l. In that case %%
%%  you may want to comment out the next line and uncomment the one %%
%%  thereafter                                                      %%
\providecommand\gnumbox{\makebox[0pt]}
%%\providecommand\gnumbox[1][]{\makebox}

%% to adjust positions in multirow situations                       %%
\setlength{\bigstrutjot}{\jot}
\setlength{\extrarowheight}{\doublerulesep}

%%  The \setlongtables command keeps column widths the same across  %%
%%  pages. Simply comment out next line for varying column widths.  %%
\setlongtables

\setlength\gnumericTableWidth{%
	53pt+%
	53pt+%
	53pt+%
0pt}
\def\gumericNumCols{3}
\setlength\gnumericTableWidthComplete{\gnumericTableWidth+%
         \tabcolsep*\gumericNumCols*2+\arrayrulewidth*\gumericNumCols}
\ifthenelse{\lengthtest{\gnumericTableWidthComplete > \linewidth}}%
         {\def\gnumericScale{1*\ratio{\linewidth-%
                        \tabcolsep*\gumericNumCols*2-%
                        \arrayrulewidth*\gumericNumCols}%
{\gnumericTableWidth}}}%
{\def\gnumericScale{1}}

%%%%%%%%%%%%%%%%%%%%%%%%%%%%%%%%%%%%%%%%%%%%%%%%%%%%%%%%%%%%%%%%%%%%%%
%%                                                                  %%
%% The following are the widths of the various columns. We are      %%
%% defining them here because then they are easier to change.       %%
%% Depending on the cell formats we may use them more than once.    %%
%%                                                                  %%
%%%%%%%%%%%%%%%%%%%%%%%%%%%%%%%%%%%%%%%%%%%%%%%%%%%%%%%%%%%%%%%%%%%%%%

\ifthenelse{\isundefined{\gnumericColA}}{\newlength{\gnumericColA}}{}\settowidth{\gnumericColA}{\begin{tabular}{@{}p{53pt*\gnumericScale}@{}}x\end{tabular}}
\ifthenelse{\isundefined{\gnumericColB}}{\newlength{\gnumericColB}}{}\settowidth{\gnumericColB}{\begin{tabular}{@{}p{53pt*\gnumericScale}@{}}x\end{tabular}}
\ifthenelse{\isundefined{\gnumericColC}}{\newlength{\gnumericColC}}{}\settowidth{\gnumericColC}{\begin{tabular}{@{}p{53pt*\gnumericScale}@{}}x\end{tabular}}

\begin{tabular}[c]{%
	b{\gnumericColA}%
	b{\gnumericColB}%
	b{\gnumericColC}%
	}

%%%%%%%%%%%%%%%%%%%%%%%%%%%%%%%%%%%%%%%%%%%%%%%%%%%%%%%%%%%%%%%%%%%%%%
%%  The longtable options. (Caption, headers... see Goosens, p.124) %%
%	\caption{The Table Caption.}             \\	%
% \hline	% Across the top of the table.
%%  The rest of these options are table rows which are placed on    %%
%%  the first, last or every page. Use \multicolumn if you want.    %%

%%  Header for the first page.                                      %%
%	\multicolumn{3}{c}{The First Header} \\ \hline 
%	\multicolumn{1}{c}{colTag}	%Column 1
%	&\multicolumn{1}{c}{colTag}	%Column 2
%	&\multicolumn{1}{c}{colTag}	\\ \hline %Last column
%	\endfirsthead

%%  The running header definition.                                  %%
%	\hline
%	\multicolumn{3}{l}{\ldots\small\slshape continued} \\ \hline
%	\multicolumn{1}{c}{colTag}	%Column 1
%	&\multicolumn{1}{c}{colTag}	%Column 2
%	&\multicolumn{1}{c}{colTag}	\\ \hline %Last column
%	\endhead

%%  The running footer definition.                                  %%
%	\hline
%	\multicolumn{3}{r}{\small\slshape continued\ldots} \\
%	\endfoot

%%  The ending footer definition.                                   %%
%	\multicolumn{3}{c}{That's all folks} \\ \hline 
%	\endlastfoot
%%%%%%%%%%%%%%%%%%%%%%%%%%%%%%%%%%%%%%%%%%%%%%%%%%%%%%%%%%%%%%%%%%%%%%

\hhline{|-|-~}
	 \multicolumn{1}{|p{\gnumericColA}|}%
	{\gnumericPB{\centering}\gnumbox{\textbf{वस्त}}}
	&\multicolumn{1}{p{\gnumericColB}|}%
	{\gnumericPB{\centering}\gnumbox{\textbf{मूल्य}}}
	&
\\
\hhline{|--|~}
	 \multicolumn{1}{|p{\gnumericColA}|}%
	{\gnumericPB{\centering}\gnumbox{अवस्था ($N$)}}
	&\multicolumn{1}{p{\gnumericColB}|}%
	{\gnumericPB{\centering}\gnumbox{10}}
	&
\\
\hhline{|--|~}
	 \multicolumn{1}{|p{\gnumericColA}|}%
	{\gnumericPB{\centering}\gnumbox{द्विविध}}
	&\multicolumn{1}{p{\gnumericColB}|}%
	{\gnumericPB{\centering}\gnumbox{$\ceil{\log_2{N}} = 4$}}
	&
\\
\hhline{|-|-|~}
\end{tabular}

\ifthenelse{\isundefined{\languageshorthands}}{}{\languageshorthands{\languagename}}
\gnumericTableEnd

}
\caption{यंत्रोपवस्तु मूल्य।}
\label{table:fsm_counter}
\end{table}

उपरोक्त विधान से पूर्ववर्ती गूढ़वाचक का अभिकल्प करें।

%\begin{equation}
%\text{No. of D Flip-Flops} = \ceil{\log_{2}\brak{\text{No. of States}}}
%\end{equation}
%For the FSM in Fig. \ref{fig:fsm_counter}, the number of states is 9, hence the number flipflops required = 4.  
%\begin{problem}
%Design a decade down counter (counts from 9 to 0 repeatedly) using an FSM.  
%\end{problem}

\end{document}


