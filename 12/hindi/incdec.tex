\renewcommand{\theequation}{\theenumi}
\renewcommand{\thefigure}{\theenumi}
\begin{enumerate}[label=\thesection.\arabic*.,ref=\thesection.\theenumi]
\numberwithin{equation}{enumi}
\numberwithin{figure}{enumi}
\numberwithin{table}{enumi}

\item परवर्ती गूढ़वाचक में आगत  दशमलव अंक  $0,1,\dots,9$  द्वयाधारी संख्या रूप में हैं एवं  परवर्ती संख्या निर्गत है । सारणी \ref{table:counter_decoder} में अनुरूप सत्य सारणी  उपलब्ध है
।
%\onecolumn
\begin{table}[!h]
\centering
%\resizebox{\columnwidth}{!}
%{
\input{./tables/counter_decoder}
%}
\caption{परवर्ती गूढ़वाचक की सत्य सारणी । }
\label{table:counter_decoder}
\end{table}
\item सारणी \ref{table:counter_decoder} में निर्गत चर $A$, आगत शब्द  $ZYXW=0000,0010,0100,0110,1000$ के लिए सत्य हैं  ।  
\item बूलीय तर्कानुसार सारणी \ref{table:counter_decoder} में निर्गत चर $A, B, C$ एवं $D$  को आगत चर $W,X,Y,Z$ के द्वारा  निम्न रूप से व्यक्त किया जा सकता है
%
\begin{align}
\label{eq:inc_A}
A &= W^{\prime}X^{\prime}Y^{\prime}Z^{\prime} + W^{\prime}XY^{\prime}Z^{\prime}
+W^{\prime}X^{\prime}YZ^{\prime}
\nonumber \\
 & \quad +W^{\prime}XYZ^{\prime}
+W^{\prime}X^{\prime}Y^{\prime}Z
\\
\label{eq:inc_B}
B &= WX^{\prime}Y^{\prime}Z^{\prime} + W^{\prime}XY^{\prime}Z^{\prime}
\nonumber \\ 
& \quad 
+WX^{\prime}YZ^{\prime}
+W^{\prime}XYZ^{\prime}
\\
\label{eq:inc_C}
C &= WXY^{\prime}Z^{\prime} + W^{\prime}X^{\prime}YZ^{\prime}
\nonumber \\ 
& \quad 
+WX^{\prime}YZ^{\prime}
+W^{\prime}XYZ^{\prime}
\\
D &= WXYZ^{\prime} + W^{\prime}X^{\prime}Y^{\prime}Z
\label{eq:inc_D}
\end{align}
\item पूरक चर की परिभाषा सारणी \ref{table:complement} में उपलब्ध है । $W$ का पूरक   $W^{\prime}$ है ।
\begin{table}[!ht]
\centering
\begin{tabular}{|l|l|}
\hline
W & $W^{\prime}$ \\ \hline
0 & 1            \\ \hline
1 & 0            \\ \hline
\end{tabular}
\caption{पूरक चर ।}
\label{table:complement}
\end{table}
\item  निम्न समीकरण बूलीय बीजगणित का मूल अभिगृह  है।
\begin{align}
\label{eq:inc_Boolean}
\begin{split}
X+X^{\prime} &= 1
\\
XX^{\prime} &= 0,
\end{split}
\end{align}

\item निम्न गूढ़ को भिन्न भिन्न आगत मूल्यों के लिये निष्पादित कर \eqref{eq:inc_A}-\eqref{eq:inc_D} को सत्यापित करें ।
\label{code:inc_decode}
\begin{lstlisting}
codes/inc_decode.c
\end{lstlisting}
%\item Modify the above C code to verify \eqref{eq:inc_A}, \eqref{eq:inc_B}
%and \eqref{eq:inc_C}.
%
%\item Repeat the exercise for the truth table in \ref{table:disp_dec}.
%%
%\begin{table}
%\centering
%\input{./tables/disp_dec.tex}
%\caption{Truth table for display decoder.}
%\label{table:disp_dec}
%\end{table}

\end{enumerate}
%
%
