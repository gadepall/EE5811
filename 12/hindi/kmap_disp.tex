निम्न  चरणों में सारणी \ref{table:disp_dec}  एवं के-मानचित्र के द्वारा $a,b,c,d,e,f,g$ के न्यूनतम व्यंजक को व्युत्पन्न किया जाएगा.

\renewcommand{\theequation}{\theenumi}
\renewcommand{\thefigure}{\theenumi}
\begin{enumerate}[label=\thesubsection.\arabic*.,ref=\thesubsection.\theenumi]
\numberwithin{equation}{enumi}
\numberwithin{figure}{enumi}
\numberwithin{table}{enumi}

\item आकृति \ref{fig:disp_kmap_a} के द्वारा $a$ के व्यंजक को व्युत्पन्न करें.


\solution

\begin{align}
\label{eq:kmap_disp_a}
a &= A B^{\prime}C^{\prime} D^{\prime} + A^{\prime}B^{\prime} C D^{\prime}
\\
&= B^{\prime}D^{\prime}\brak{A C^{\prime}+ A^{\prime}C}
\\
&=B^{\prime}D^{\prime}\brak{A \oplus  C}
\end{align}
%
\begin{figure}[!ht]
\centering
\resizebox{\columnwidth}{!} {

\begin{karnaugh-map}[4][4][1][][]
    \maxterms{0,2,3,5,6,7,8,9}

    \minterms{1,4}
%    \implicantedge{1}{1}{9}{9}
    \implicant{4}{4}
    \implicant{1}{1}
    
    % note: position for start of \draw is (0, Y) where Y is
    % the Y size(number of cells high) in this case Y=2
    \draw[color=black, ultra thin] (0, 4) --
    node [pos=0.7, above right, anchor=south west] {$BA$} % YOU CAN CHANGE NAME OF VAR HERE, THE $X$ IS USED FOR ITALICS
    node [pos=0.7, below left, anchor=north east] {$DC$} % SAME FOR THIS
    ++(135:1);
        
    \end{karnaugh-map}

}
\caption{$a$ का क-मानचित्र.}
\label{fig:disp_kmap_a}
\end{figure}
%
$\oplus$ संक्रिया की परिभाषा सारणी \ref{table:xor}  में उपलब्ध     है.

\begin{table}[!ht]
\centering
\begin{tabular}{|l|l|l|}
\hline
$A$ & $C$ & $A\oplus C$ \\ \hline
0   & 0   & 0          \\ \hline
0   & 1   & 1          \\ \hline
1   & 0   & 1          \\ \hline
1   & 1   & 0          \\ \hline
\end{tabular}
\caption{$\oplus$ की परिभाषा.}
\label{table:xor}
\end{table}

\item आकृति \ref{fig:disp_kmap_b} के द्वारा $b$ के व्यंजक को व्युत्पन्न करें.


\solution

\begin{align}
\label{eq:kmap_disp_b}
b &= A B^{\prime}C D^{\prime} + A^{\prime}B C D^{\prime}
\\
&=C D^{\prime}\brak{A \oplus B}
\end{align}
%


\begin{figure}[!ht]
\centering
\resizebox{\columnwidth}{!} {
\begin{karnaugh-map}[4][4][1][][]
    \maxterms{0,1,2,3,4,7,8,9}
\terms{10,11,12,13,14,15}{X}
    \minterms{5,6}
     \implicant{6}{14}    
     \implicant{5}{13}    
%     \implicant{10}{10}
    % note: position for start of \draw is (0, Y) where Y is
    % the Y size(number of cells high) in this case Y=2
    \draw[color=black, ultra thin] (0, 4) --
    node [pos=0.7, above right, anchor=south west] {$BA$} % Y label
    node [pos=0.7, below left, anchor=north east] {$DC$} % X label
    ++(135:1);
        
    \end{karnaugh-map}

}
\caption{$b$ का क-मानचित्र।}
\label{fig:disp_kmap_b}
\end{figure}
%
\item आकृति \ref{fig:disp_kmap_c} के द्वारा $c$ के व्यंजक को व्युत्पन्न करें।


\solution

\begin{align}
\label{eq:kmap_disp_c}
c &=  A^{\prime}B C^{\prime} D^{\prime}
\end{align}
%
\begin{figure}[!ht]
\centering
\resizebox{\columnwidth}{!} {

\begin{karnaugh-map}[4][4][1][][]
    \maxterms{1,4,0,3,5,6,7,8,9}
%    \terms{}{X}
%    \terms{10,11,12,13,14,15}{X}
    \minterms{2}
%    \implicantedge{1}{1}{9}{9}
%    \implicant{4}{4}
    \implicant{2}{2}
    
    % note: position for start of \draw is (0, Y) where Y is
    % the Y size(number of cells high) in this case Y=2
    \draw[color=black, ultra thin] (0, 4) --
    node [pos=0.7, above right, anchor=south west] {$BA$} % YOU CAN CHANGE NAME OF VAR HERE, THE $X$ IS USED FOR ITALICS
    node [pos=0.7, below left, anchor=north east] {$DC$} % SAME FOR THIS
    ++(135:1);
        
    \end{karnaugh-map}

}
\caption{$c$ का क-मानचित्र।}
\label{fig:disp_kmap_c}
\end{figure}
%
\item  आकृति \ref{fig:disp_kmap_d} के द्वारा $d$ के व्यंजक को व्युत्पन्न करें।
\\
\solution


\begin{align}
\label{eq:kmap_disp_d}
d=AB^{\prime}C^{\prime}+A^{\prime}B^{\prime}CD^{\prime}+ABCD^{\prime}
\end{align}
%
\begin{figure}[!ht]
\centering
\resizebox{\columnwidth}{!} {

\begin{karnaugh-map}[4][4][1][][]
    \maxterms{0,2,3,5,6,8}
\terms{10,11,12,13,14,15}{X}
    \minterms{1,4,7,9}
    \implicantedge{1}{1}{9}{9}
    \implicant{4}{12}
    \implicant{7}{15}
    
    % note: position for start of \draw is (0, Y) where Y is
    % the Y size(number of cells high) in this case Y=2
    \draw[color=black, ultra thin] (0, 4) --
    node [pos=0.7, above right, anchor=south west] {$BA$} % YOU CAN CHANGE NAME OF VAR HERE, THE $X$ IS USED FOR ITALICS
    node [pos=0.7, below left, anchor=north east] {$DC$} % SAME FOR THIS
    ++(135:1);
        
    \end{karnaugh-map}

}
\caption{$d$ का क-मानचित्र।}
\label{fig:disp_kmap_d}
\end{figure}
\item आकृति \ref{fig:disp_kmap_e} के द्वारा $e$ के व्यंजक को व्युत्पन्न करें।
%

\begin{align}
\label{eq:kmap_disp_e}
e=AD^{\prime}+B^{\prime}CD^{\prime}+AB^{\prime}C^{\prime}
\end{align}
%
\begin{figure}[!ht]
\centering
\resizebox{\columnwidth}{!} {
\usetikzlibrary{arrows,shapes.gates.logic.US,shapes.gates.logic.IEC,calc}
\begin{tikzpicture}[label distance=5mm]

    \node (x3) at (0,0) {$A$};
    \node (x2) at (1,0) {$D$};
    \node (x1) at (2,0) {$B$};
    \node (x0) at (3,0) {$C$};

    \node[not gate US, draw, rotate=-90] at ($(x2)+(0,-1)$) (Not2) {};
    \node[not gate US, draw, rotate=-90] at ($(x1)+(0,-1)$) (Not1) {};
    \node[not gate US, draw, rotate=-90] at ($(x0)+(0.5,-1)$) (Not0) {};

    \node[and gate US, draw, logic gate inputs=nnnn] at ($(x0)+(2,-2)$) (Or1) {};
    \node[and gate US, draw, logic gate inputs=nnnn] at ($(Or1)+(0,-1)$) (Or2) {};
    \node[and gate US, draw, logic gate inputs=nnnn
] at ($(Or2)+(0,-1)$) (Or3) {};
  
     \node[or gate US, draw, logic gate inputs=nnnn, anchor=input 1,2,3] at ($(Or1.output)+(1,0)$) (And1) {};
     
    \draw(0,-0.2)--(0,-1.7)--(4.5,-1.7);
    \draw(1,-0.2)--(1,-0.5);
    \draw(1,-1.4)--(1,-2.2)--(4.5,-2.2);
   \draw(0,-1.7)--(0,-3)--(4.5,-3);
   \draw(2,-0.2)--(2,-0.5);
   \draw(2,-1.4)--(2,-2.7)--(4.5,-2.7);
   \draw(3,-0.2)--(3,-0.49);
   \draw(3.5,-1.4)--(3.5,-3.3)--(4.5,-3.3);
   \draw(2,-2.7)--(2,-4)--(4.5,-4);
   \draw(3,-0.49)--(3,-3.7)--(4.5,-3.7);
   \draw(1,-2.2)--(1,-4.3)--(4.5,-4.3);
   \draw(5.6,-2)--(6.6,-2);
   \draw(5.6,-3)--(6,-3)--(6,-2.3)--(6.6,-2.3);
   \draw(5.6,-4)--(6.2,-4)--(6.2,-2.5)--(6.6,-2.5);
   \draw(7.64,-2.25)--(8.5,-2.25);
 
   
   
   

    \foreach \i in {2,1,0}
    {
        \path (x\i) -- coordinate (punt\i) (x\i |- Not\i.input);
        \draw (punt\i) node[branch] {} -| (Not\i.input);
    }
   

\end{tikzpicture}


}
\caption{$e$ का क-मानचित्र।}
\label{fig:disp_kmap_e}
\end{figure}
%
\item आकृति \ref{fig:disp_kmap_f} के द्वारा $f$ के व्यंजक को व्युत्पन्न करें।
%
\begin{align}
\label{eq:kmap_disp_f}
f= BC^{\prime}D^{\prime} + AB^{\prime}C^{\prime}D^{\prime} + AB^{\prime}CD
\end{align}
%
\begin{figure}[!ht]
\centering
\resizebox{\columnwidth}{!} {
\begin{karnaugh-map}[4][4][1][][]
    \maxterms{0,4,5,6,8,9}
%\terms{10,11,12,13,14,15}{X}
    \minterms{1,2,3,7}
%    \implicant{11}{11}
%\implicantedge{3}{2}{11}{10}
\implicant{1}{3}
   \implicant{3}{2}
   \implicant{3}{7}
    %\implicant{8}{8}
    % note: position for start of \draw is (0, Y) where Y is
    % the Y size(number of cells high) in this case Y=2
    \draw[color=black, ultra thin] (0, 4) --
    node [pos=0.7, above right, anchor=south west] {$BA$} % YOU CAN CHANGE NAME OF VAR HERE, THE $X$ IS USED FOR ITALICS
    node [pos=0.7, below left, anchor=north east] {$DC$} % SAME FOR THIS
    ++(135:1);
        
    \end{karnaugh-map}

}
\caption{$f$ का क-मानचित्र।}
\label{fig:disp_kmap_f}
\end{figure}

%
\item आकृति \ref{fig:disp_kmap_g} के द्वारा $g$ के व्यंजक को व्युत्पन्न करें।
%
\begin{align}
\label{eq:kmap_disp_g}
g = B^{\prime}C^{\prime}D^{\prime}+ABCD^{\prime}
\end{align}
%
\begin{figure}[!ht]
\centering
\resizebox{\columnwidth}{!} {
\begin{karnaugh-map}[4][4][1][][]
    \maxterms{2,3,4,5,6,8,9}
\terms{10,11,12,13,14,15}{X}
    \minterms{0,1,7}

    \implicant{0}{1}
    \implicant{7}{15}
    
    % note: position for start of \draw is (0, Y) where Y is
    % the Y size(number of cells high) in this case Y=2
    \draw[color=black, ultra thin] (0, 4) --
    node [pos=0.7, above right, anchor=south west] {$BA$} % YOU CAN CHANGE NAME OF VAR HERE, THE $X$ IS USED FOR ITALICS
    node [pos=0.7, below left, anchor=north east] {$DC$} % SAME FOR THIS
    ++(135:1);
  
      
    \end{karnaugh-map}

}
\caption{$g$ का क-मानचित्र।}
\label{fig:disp_kmap_g}
\end{figure}
\end{enumerate}
%
%
